\chapter{\ifcpe บทนำ\else Introduction\fi}

\section{\ifcpe ที่มาของโครงงาน\else Project rationale\fi}


การจัดตารางสอบสำหรับทุกรายวิชาที่เปิดสอนภายในมหาวิทยาลัยนั้น
เป็นเรื่องยากที่จะจัดการไม่ให้เกิดการ\CI{ทับซ้อนกัน}{แปลว่าอะไร}ของบางรายวิชาเกิดขึ้น
ทางสำนักทะเบียนจึงใช้วิธีการจัดตารางสอบโดยการแบ่งรายวิชาที่เปิดสอนเพียงหนึ่งตอนเป็นวิชาปกติ
และวิชาที่เปิดสอนมากกว่าหนึ่งตอนให้เป็นวิชาพิเศษ
โดยการจัดตารางสอบปลายภาคของวิชาปกติที่มีเวลาเรียนเหมือนกันจะถูกจัดให้สอบในช่วงเวลาเดียวกัน
โดยอ้างอิงจากช่วงเวลาที่จัดการเรียนการสอน ส่วนวิชาพิเศษจะเลือกช่วงเวลาในการสอบช่วงใดก็ได้ในระยะเวลาที่จัดสอบ
\CIreply{กำลังจะขึ้น point ใหม่ แต่อ่านมาแล้วยังไม่แน่ใจว่า point เดิมต้องการจะสื่ออะไร}
\CIreply{อาจจะสรุปเป็นประเด็นไปว่า ที่ยกมานั้นเกิดปัญหาอะไรบ้าง}
อีกทั้งตารางสอบที่ใช้อยู่ในปัจจุบันถูกจัดให้แต่ละวันมีช่วงเวลาในการสอบ 3 ช่วงเวลา
จึงทำให้นักศึกษาบางคนอาจจะต้องสอบถึง 3 รายวิชาในหนึ่งวัน 
\CIreply{เปลี่ยน point ตรงนี้อีกแล้ว แต่ยังไม่ได้สรุป}
และเนื่องจากตารางสอบของแต่ละปีนั้นใช้งานได้อยู่แล้ว
จึงทำให้ทางสำนักทะเบียนไม่ได้เปลี่ยนวิธีการจัดตารางสอบ ซึ่งทำให้แต่ละรายวิชาที่มีการเปิดสอนมักจะถูกจัดให้สอบในช่วงเวลาเดิม ๆ
\CI{ทำให้นักศึกษาไม่สามารถลงทะเบียนเรียนบางรายวิชาพร้อมกันได้ เนื่องจากคู่รายวิชานั้นจะต้องถูกจัดให้สอบในช่วงเวลาเดียวกันตามตารางสอบของปีที่ผ่าน ๆ มา}{point นี้สำคัญ ต้องย้ำบ่อยๆ ในจุดอื่นที่เกี่ยวข้อง}
หรือในบางรายวิชาที่เป็นวิชาบังคับที่นักศึกษาจะต้องลงคู่กันอาจจะมีบางคู่วิชาที่ถูกจัดให้สอบต่อกันในวันเดียวกัน
การจัดตารางสอบด้วยระบบแบบนี้จึงเป็นการผลักภาระให้กับนักศึกษาต้องถูกบังคับให้เลือกลงทะเบียนเรียนเฉพาะวิชาที่ตารางสอบไม่ทับซ้อนกันเท่านั้น 
\enskip
ทางผู้จัดทำจึงได้จัดทำระบบจัดตารางสอบปลายภาค ซึ่งเป็นระบบที่ช่วยจัดการปัญหาการจัดตารางสอบของนักศึกษา
ทุกระดับในมหาวิทยาลัย ซึ่งจะช่วยให้บางคู่รายวิชาที่เปิดสอนไม่จำเป็นต้องถูกบังคับให้สอบตรงกันเสมอหากมีนักศึกษาที่ต้องการลงทะเบียนเรียนทั้งสองวิชานี้พร้อม ๆ กัน
ทำให้นักศึกษาทุกคนสามารถเลือกเรียนวิชาที่ตนเองสนใจในแต่ละเทอมได้ โดยไม่มีข้อจำกัดด้วยช่วงเวลาสอบที่ทับซ้อนกัน
โดยระบบจะพยายามจัดช่วงเวลาสอบของแต่ละรายวิชาให้เหมาะสมกับนักศึกษาทุกคนมากที่สุดเท่าที่จะเป็นไปได้ กล่าวคือ ตารางสอบของนักศึกษาคนใด ๆ ก็ตาม 
\CI{ไม่ควรให้ในหนึ่งวันมีการจัดสอบถึงสามรายวิชาที่นักศึกษาลงทะเบียน}{``นักศึกษาไม่ควรจำเป็นจะต้องสอบถึงสามวิชาในวันเดียวกัน''? \\ น่าจะต้องไปพูดละเอียดใน objective อีกรอบ ไม่แน่ใจว่าจำเป็นต้องพูดละเอียดตรงนี้หรือยัง} และสำหรับวันที่มี\CI{การจัดสอบสองรายวิชานั้น}{ต้องพูดให้เคลียร์ว่าเป็นตารางของนักศึกษาคนใดคนหนึ่ง แต่โดยรวมอาจจะจัดสอบทั้งสามช่วงเวลาได้อยู่ดี} จะพยายามไม่ให้สองรายวิชาอยู่ในช่วงเวลาที่ติดกัน
โดยหากเป็นไปได้จะจัดตารางสอบให้เหลือเพียงวันละ 2 ช่วงเวลา ได้แก่ ช่วงเช้าและช่วงเย็นเท่านั้น
\CIreply{แยกเขียนหลายย่อหน้าได้ เพราะมันอาจจะยาวได้หลายหน้าอยู่}

\section{\ifcpe วัตถุประสงค์ของโครงงาน\else Objectives\fi}
\begin{enumerate}
    \item เพื่อออกแบบอัลกอริทึมสำหรับจัดตารางสอบปลายภาคโดยสามารถกำหนดข้อจำกัดต่าง ๆ ได้ ?\CIreply{ได้แก่?}
    \item เพื่อแก้ไขปัญหาตารางสอบปลายภาคของนักศึกษาที่มีความหนาแน่นมากเกินไป
    \item เพื่อสร้างโปรแกรมสำหรับจัดตารางสอบปลายภาค ? \CIreply{ต่างจากอันแรกอย่างไร}
    \item เพื่อออกแบบข้อจำกัดในการจัดตารางสอบปลายภาคเพื่อใช้งานกับ Solver ? \CIreply{ต่างจากอันแรกอย่างไร}
\end{enumerate}
\CIreply{พูดถึงคุณภาพของ solution ตรงนี้?}

\section{\ifcpe ขอบเขตของโครงงาน\else Project scope\fi}

\subsection{\ifcpe ขอบเขตด้านฮาร์ดแวร์\else Hardware scope\fi}
\begin{enumerate}
    \item คอมพิวเตอร์ที่มีความสามารถในการประมวลผลสูง เพื่อลดเวลาในการประมวลผลตารางสอบ
\end{enumerate}
\subsection{\ifcpe ขอบเขตด้านซอฟต์แวร์\else Software scope\fi}
\CIreply{real-time? เวลาที่ใช้?}
\subsection{ข้อจำกัดที่นำมาพิจารณาในการจัดตารางสอบ}
\begin{enumerate}
    \item จำนวนนักศึกษาที่ลงทะเบียนแต่ละวิชาเรียน
    \item จำนวนที่นั่งในแต่ละห้องสอบของแต่ละคณะ
    \item จำนวนวิชาเรียนที่มีการจัดสอบปลายภาค
    \item คู่วิชาเรียนที่มีนักศึกษาลงทะเบียนร่วมกันทั้งสองวิชา
    \item จำนวนวันที่ทำการจัดสอบ
\end{enumerate}

\subsection{ข้อจำกัดที่ไม่นำมาพิจารณาในการจัดตารางสอบ}
\begin{enumerate}
    \item อาจารย์ผู้คุมสอบ\CIreply{อธิบาย?}
    \item เวลาที่จัดการเรียนการสอนของแต่ละรายวิชา
\end{enumerate}

\section{\ifcpe ประโยชน์ที่ได้รับ\else Expected outcomes\fi}
\begin{enumerate}
    \item ช่วยให้การจัดตารางสอบปลายภาคในระดับมหาวิทยาลัยมีความ\CI{สะดวกมากขึ้น}{ตอนนี้ไม่สะดวกอย่างไร?}
    \item นักศึกษาสามารถเลือกลงทะเบียนวิชาที่ต้องการได้โดยไม่มีข้อจำกัดเรื่องตารางสอบทับซ้อนกัน
    \item ตารางสอบของนักศึกษามีความยืดหยุ่นมากขึ้น\CIreply{อย่างไร}
    \item สามารถจัดตารางสอบได้รวดเร็วมากขึ้นหากข้อจำกัดด้านพื้นที่มีการเปลี่ยนแปลง ?\CIreply{อธิบายเพิ่มเติม?}
    \item ช่วยลดจำนวนวันที่ใช้ในการจัดการสอบ\CIreply{แน่ใจหรือเปล่า?}
\end{enumerate}
\CIreply{น่าจะต้องปรับเป็นร้อยแก้วทีหลัง}

\section{\ifcpe เทคโนโลยีและเครื่องมือที่ใช้\else Technology and tools\fi}

\subsection{\ifcpe เทคโนโลยีด้านฮาร์ดแวร์\else Hardware technology\fi}
\begin{enumerate}
    \item Cloud Computing ?
\end{enumerate}
\subsection{\ifcpe เทคโนโลยีด้านซอฟต์แวร์\else Software technology\fi}
\begin{enumerate}
    \item Google OR-Tools ?
\end{enumerate}

\section{\ifcpe แผนการดำเนินงาน\else Project plan\fi}

\begin{plan}{7}{2020}{3}{2021}
    \planitem{7}{2020}{8}{2020}{Literature Review}
    \planitem{8}{2020}{9}{2020}{เขียนอัลกอริทึม เวอร์ชัน 1}
    \planitem{8}{2020}{10}{2020}{เก็บข้อมูลครั้งที่ 1}
    \planitem{9}{2020}{10}{2020}{เขียนอัลกอริทึม เวอร์ชัน 2}
    \planitem{10}{2020}{10}{2020}{เขียนรายงาน}
    \planitem{11}{2020}{12}{2020}{เก็บข้อมูลครั้งที่ 2}
    \planitem{11}{2020}{1}{2021}{เขียนอัลกอริทึม เวอร์ชัน 3}
    \planitem{1}{2021}{2}{2021}{ทดสอบและประเมินผลอัลกอริทึม}
    \planitem{1}{2021}{2}{2021}{ทดสอบความพึงพอใจของนักศึกษา}
    \planitem{2}{2021}{3}{2021}{เขียนรายงาน}
\end{plan}

\section{\ifcpe บทบาทและความรับผิดชอบ\else Roles and responsibilities\fi}
อธิบายว่าในการทำงาน นศ. มีการกำหนดบทบาทและแบ่งหน้าที่งานอย่างไรในการทำงาน จำเป็นต้องใช้ความรู้ใดในการทำงานบ้าง

\section{\ifcpe%
ผลกระทบด้านสังคม สุขภาพ ความปลอดภัย กฎหมาย และวัฒนธรรม
\else%
Impacts of this project on society, health, safety, legal, and cultural issues
\fi}

แนวทางและโยชน์ในการประยุกต์ใช้งานโครงงานกับงานในด้านอื่นๆ รวมถึงผลกระทบในด้านสังคมและสิ่งแวดล้อมจากการใช้ความรู้ทางวิศวกรรมที่ได้
