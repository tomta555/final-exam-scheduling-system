\chapter{\ifcpe ทฤษฎีที่เกี่ยวข้อง\else Background Knowledge and Theory\fi}
การทำโครงงาน เริ่มต้นด้วยการศึกษาค้นคว้า ทฤษฎีที่เกี่ยวข้อง หรือ งานวิจัย/โครงงาน ที่เคยมีผู้นำเสนอไว้แล้ว ซึ่งเนื้อหาในบทนี้ก็จะเกี่ยวกับการอธิบายถึงสิ่งที่เกี่ยวข้องกับโครงงาน เพื่อให้ผู้อ่านเข้าใจเนื้อหาในบทถัดๆ ไปได้ง่ายขึ้น
\section{Literature review}
การกำหนดเวลาสอบปลายภาคเพื่อหลีกเลี่ยงปัญหานักศึกษาคนใด ๆ มีเวลาสอบในช่วงเวลาเดียวกันสามารถแปลงปัญหานี้ให้เป็นปัญหา graph coloring~\cite{mcs} ได้ 
โดยที่จุดแต่ละจุดในกราฟเป็นรายวิชาที่เปิดสอนในภาคการศึกษานั้น 
และเส้นที่เชื่อมแต่ละจุดสองจุดในกราฟแสดงถึงการมีนักศึกษาที่ลงทะเบียนเรียนทั้งสองรายวิชา โดยจุดสองจุดใด ๆ ในกราฟที่มีเส้นเชื่อมกันจะถูกกำหนดสีให้ต่างกัน
ซึ่งแสดงถึงวันและเวลาที่จัดสอบปลายภาคในวิชานั้น ๆ โดยจุดที่มีคนละสีก็จะถูกจัดให้สอบคนละช่วงเวลากัน

การแปลงปัญหาการจัดตารางสอบปลายภาคให้เป็นปัญหา graph coloring จะสามารถแก้ปัญหาการจัดตารางสอบแบบพื้นฐานได้เท่านั้น โดยไม่คำนึงถึงข้อจำกัดอย่างอื่น 
ตัวอย่างเช่น ไม่พิจารณาความจุที่นั่งสำหรับการสอบแต่ละช่วงเวลาของนักศึกษา จำนวนอาจารย์ที่คุมสอบแต่ละช่วงเวลา และการกระจายวิชาสอบสำหรับนักศึกษาแต่ละคน เป็นต้น
ถึงแม้จะไม่กำหนดข้อจำกัดใด ๆ ปัญหา graph coloring ก็เป็นปัญหา NP-complete ด้วยตัวมันเองอยู่แล้ว~\cite{alg-design} 
ซึ่งหมายความว่ายังไม่สามารถหาอัลกอริทึมที่ใช้เวลา polynomial-time ในการแก้ไขปัญหาให้ได้ผลลัพธ์ที่เหมาะสมที่สุด 
ทำให้ต้องใช้วิธีอื่นที่ให้ผลลัพธ์ที่ดีในระดับที่ยอมรับได้ แต่สามารถยืนยันได้ว่าจะได้วิธีการที่สามารถแก้ไขปัญหาได้อย่างแน่นอน 
ซึ่งวิธีการนั้นคือ metaheuristic ซึ่งสามารถหาวิธีการแก้ปัญหาที่ดีได้ในระยะเวลาที่เหมาะสม \cite{meta-for-vertexcolor}
และสามารถกำหนดข้อจำกัดหรือเงื่อนไขอื่นเพิ่มเติมได้ ทำให้สามารถกำหนดขอบเขตของผลลัพธ์ได้ แต่อาจจะไม่ได้วิธีแก้ปัญหาที่ดีที่สุด
\section{Tools}
\subsection{Gurobi Optimizer}
เป็น Solver ที่ใช้สำหรับแก้ปัญหา Optimization โดยที่จะเน้นไปทางด้านของปัญหาต่าง ๆ ดังนี้ 
\begin{itemize}
  \item Linear programming (LP)
  \item Mixed-integer linear programming (MILP)
  \item Quadratic programming (QP)
  \item Mixed-integer quadratic programming (MIQP)
  \item Quadratically-constrained programming (QCP)
  \item Mixed-integer quadratically-constrained programming (MIQCP)
\end{itemize}
ผลลัพธ์ที่ได้จาก Gurobi Optimizer อาจนำมาใช้เป็นตัวเปรียบเทียบประสิทธิภาพกับผลลัพธ์การทำงานที่ได้จากอัลกอลิทึมของเรา

\section{Algorithm}
\subsection{Metaheuristic Algorithms}
\subsection{Genetic Algorithms}


\section{Second section}
Section 2 text.

\subsection{Subsection heading goes here}

Subsection 1 text

\subsubsection{Subsubsection 1 heading goes here}
Subsubsection 1 text

\subsubsection{Subsubsection 2 heading goes here}
Subsubsection 2 text

\section{Third section}
Section 3 text. The dielectric constant\index{dielectric constant}
at the air-metal interface determines
the resonance shift\index{resonance shift} as absorption or capture occurs
is shown in Equation~\eqref{eq:dielectric}:

\begin{equation}\label{eq:dielectric}
k_1=\frac{\omega}{c({1/\varepsilon_m + 1/\varepsilon_i})^{1/2}}=k_2=\frac{\omega
\sin(\theta)\varepsilon_\mathit{air}^{1/2}}{c}
\end{equation}

\noindent
where $\omega$ is the frequency of the plasmon, $c$ is the speed of
light, $\varepsilon_m$ is the dielectric constant of the metal,
$\varepsilon_i$ is the dielectric constant of neighboring insulator,
and $\varepsilon_\mathit{air}$ is the dielectric constant of air.

\section{About using figures in your report}

% define a command that produces some filler text, the lorem ipsum.
\newcommand{\loremipsum}{
  \textit{แม้งเอ้ย}\par}

\begin{figure}
  \centering

  \fbox{
     \parbox{.6\textwidth}{\loremipsum}
  }

  % To include an image in the figure, say myimage.pdf, you could use
  % the following code. Look up the documentation for the package
  % graphicx for more information.
  % \includegraphics[width=\textwidth]{myimage}

  \caption[Sample figure]{This figure is a sample containing \gls{lorem ipsum},
  showing you how you can include figures and glossary in your report.
  You can specify a shorter caption that will appear in the List of Figures.}
  \label{fig:sample-figure}
\end{figure}

Using \verb.\label. and \verb.\ref. commands allows us to refer to
figures easily. If we can refer to Figures
\ref{fig:walrus} and \ref{fig:sample-figure} by name in the {\LaTeX}
source code, then we will not need to update the code that refers to it
even if the placement or ordering of the figures changes.

\loremipsum\loremipsum

% This code demonstrates how to get a landscape table or figure. It
% uses the package lscape to turn everything but the page number into
% landscape orientation. Everything should be included within an
% \afterpage{ .... } to avoid causing a page break too early.
\afterpage{
  \begin{landscape}
  \begin{table}
    \caption{Sample landscape table}
    \label{tab:sample-table}

    \centering

    \begin{tabular}{c||c|c}
        Year & A & B \\
        \hline\hline
        1989 & 12 & 23 \\
        1990 & 4 & 9 \\
        1991 & 3 & 6 \\
    \end{tabular}
  \end{table}
  \end{landscape}
}

\loremipsum\loremipsum\loremipsum

\section{Overfull hbox}

When the \verb.semifinal. option is passed to the \verb.cpecmu. document class,
any line that is longer than the line width, i.e., an overfull hbox, will be
highlighted with a black solid rule:
\begin{center}
\begin{minipage}{2em}
juxtaposition
\end{minipage}
\end{center}

\section{\ifcpe%
ความรู้ตามหลักสูตรซึ่งถูกนำมาใช้หรือบูรณาการในโครงงาน
\else%
ISNE knowledge used, applied, or integrated in this project
\fi
}

อธิบายถึงความรู้ และแนวทางการนำความรู้ต่างๆ ที่ได้เรียนตามหลักสูตร ซึ่งถูกนำมาใช้ในโครงงาน

\section{\ifcpe%
ความรู้นอกหลักสูตรซึ่งถูกนำมาใช้หรือบูรณาการในโครงงาน
\else%
Extracurricular knowledge used, applied, or integrated in this project
\fi
}

อธิบายถึงความรู้ต่างๆ ที่เรียนรู้ด้วยตนเอง และแนวทางการนำความรู้เหล่านั้นมาใช้ในโครงงาน
