\chapter{\ifproject%
\ifcpe โครงสร้างและขั้นตอนการทำงาน\else Project Structure and Methodology\fi
\else%
\ifcpe โครงสร้างของโครงงาน\else Project Structure\fi
\fi
}


\makeatletter

% \renewcommand\section{\@startsection {section}{1}{\z@}%
%                                    {13.5ex \@plus -1ex \@minus -.2ex}%
%                                    {2.3ex \@plus.2ex}%
%                                    {\normalfont\large\bfseries}}

\makeatother
%\vspace{2ex}
% \titleformat{\section}{\normalfont\bfseries}{\thesection}{1em}{}
% \titlespacing*{\section}{0pt}{10ex}{0pt}

\section{การสำรวจความคิดเห็นของนักศึกษาเกี่ยวกับตารางสอบปลายภาค}
จากความต้องการพัฒนาระบบจัดตารางสอบปลายภาคให้นักศึกษามีอิสระในการเลือกลงวิชาเรียนมากขึ้นที่กล่าวในตอนที่ \ref{sec:Project rationale} เราจึงต้องการทราบความคิดเห็นของนักศึกษาในมหาวิทลัยเชียงใหม่เกี่ยวกับตารางสอบปลายภาค 
ทำให้เราได้จัดทำแบบสอบถามขึ้นเพื่อขอความคิดเห็นของนักศึกษาในมหาวิทาลัยเชียงใหม่ที่ทำการศึกษาอยู่และสำเร็จการศึกษาแล้ว ว่าคิดอย่างไรกับการจัดตารางสอบปลายภาคในปัจจุบันและช่วงเวลาที่นักศึกษาส่วนใหญ่นั้นต้องการที่จะสอบ 
โดยจะพยายามนำข้อมูลที่ได้มาช่วยในการประมวลผลหาวิธีการจัดตารางสอบที่เหมาะสมที่สุดที่เป็นไปได้สำหรับทุกคน
\CIreply{overview ก่อนว่าถามคำถามอะไรบ้าง หรือจุดประสงค์ของการเก็บข้อมูลคืออะไร เพื่อให้เห็นภาพรวมก่อนลงรายละเอียดเกี่ยวกับผลสำรวจ}

จากผลสำรวจของแบบสอบถามเกี่ยวกับตารางสอบปลายภาคของมหาวิทยาลัยชียงใหม่ โดยขอความร่วมมือนักศึกษาในมหาวิทยาลัย
ทั้งนักศึกษาที่กำลังศึกษาอยู่และทั้งที่สำเร็จการศึกษาไปแล้ว เพื่อให้ช่วยตอบแบบสอบถามความคิดเห็นเกี่ยวกับ ข้อดี ข้อเสีย ความพึงพอใจในตารางสอบของตนเอง
รวมถึงปัญหาเกี่ยวกับตารางสอบปลายภาคที่เคยพบหรือได้รับผลกระทบโดยตรง โดยจากผลการสำรวจกลุ่มสำรวจจำนวน 95 คน พบว่าผู้ตอบแบบสอบถามส่วนใหญ่นั้นตรวจสอบตารางสอบของทุกวิชาที่ต้องการจะลงทะเบียน
ก่อนที่จะลงทะเบียนเรียนอย่างสม่ำเสมอ แต่ยังมีผู้ตอบแบบสอบถามบางส่วนนั้นที่ตรวจสอบตารางสอบเพียงบางวิชาก่อนจะลงทะเบียนเรียน และยังมีมีผู้ตอบแบบสอบถามส่วนน้อยที่ตอบว่าไม่เคยตรวจสอบตารางสอบของตนเองเลย ดังกราฟที่ \ref{fig:check_before_enrollment}
\begin{figure}
  \begin{center}
    \includegraphics{images/checking_schedule_before_enrollment.png}\\[2ex]
    \includegraphics{images/legend_for_checking_schedule_before_enrollment.png}
  \end{center}
  \caption[Poem]{จำนวนผู้ตอบแบบสอบถามที่ตรวจสอบตารางสอบปลายภาคก่อนการลงทะเบียน}
  \label{fig:check_before_enrollment}     
\end{figure}
จากผลการสำรวจเรายังพบว่ากลุ่มสำรวจกว่า 80\% ไม่ทราบว่าสำนักทะเบียนมหาลัยเชียงใหม่ จัดตารางสอบปลายภาคอย่างไร ดังแสดงในกราฟที่ \ref{fig:registration_exam}
\begin{figure}
  \begin{center}
    \includegraphics{images/registration_exam.png}\\[2ex]
    \includegraphics{images/legend_for_registration_exam.png}
  \end{center}
  \caption[Poem]{จำนวนผู้ตอบแบบสอบถามที่ทราบวิธีการจัดตารางสอบปลายภาคของสำนักทะเบียน}
  \label{fig:registration_exam}     
\end{figure}
จากผลการสำรวจเรายังสามารถสรุปผลได้ดังนี้
วันที่ผู้ทำแบบสอบถามต้องการจะสอบมากที่สุด 7 อันดับแรก โดยเรียงลำดับจากมากไปน้อย
\begin{enumerate}
  \item สัปดาห์ที่หนึ่ง วันจันทร์
  \item สัปดาห์ที่หนึ่ง วันพุธ
  \item สัปดาห์ที่หนึ่ง วันศุกร์ 
  \item สัปดาห์ที่หนึ่ง วันอาทิตย์
  \item สัปดาห์ที่สอง วันอังคาร
  \item สัปดาห์ที่สอง วันพฤหัสบดี
  \item สัปดาห์ที่สอง วันเสาร์
\end{enumerate}

\noindent จากข้อมูลเราสามารถบอกได้ว่าเวลาที่ผู้ทำแบบสอบถามส่วนใหญ่ต้องการที่จะสอบในแต่ละวันคือ
ช่วงเวลา 12.00-15.00น. ซึ่งผู้ทำแบบสอบถามส่วนมากต้องการจะสอบช่วงเวลานี้มากกว่า 15.30-18.00น. และ 08.00-11.00น. ตามลำดับ ดังกราฟ \ref{fig:time}
\begin{figure}
  \begin{center}
    \includegraphics[width=\linewidth]{images/pie_chart_for_final_exam_time.png}
  \end{center}
  \caption[Poem]{ความต้องการในสอบของผู้ตอบแบบสอบถามในแต่ละเวลา}
  \label{fig:time}     
\end{figure}
และถ้าเราทำการรวมช่วงเวลาที่ผู้ตอบแบบสอบถามต้องการสอบกับวันที่ผู้ตอบแบบสอบถามต้องการสอบเข้าด้วยกันดังกราฟที่ \ref{fig:time_slot} จะสามารถสรุปวันที่ผู้ทำแบบสอบถามต้องการจะสอบมากที่สุด 7 อันดับแรก โดยสามารถเรียงลำดับจากมากไปน้อยได้ ดังนี้
\begin{enumerate}
  \item สัปดาห์ที่หนึ่ง เวลา 12.00-15.00น. วันศุกร์ 
  \item สัปดาห์ที่หนึ่ง เวลา 12.00-15.00น. วันจันทร์
  \item สัปดาห์ที่หนึ่ง เวลา 12.00-15.00น. วันพุธ
  \item สัปดาห์ที่สอง เวลา 12.00-15.00น. วันอังคาร
  \item สัปดาห์ที่หนึ่ง เวลา 12.00-15.00น. วันอาทิตย์
  \item สัปดาห์ที่สอง เวลา 12.00-15.00น. วันพฤหัสบดี
  \item สัปดาห์ที่สอง เวลา 12.00-15.00น. วันเสาร์
\end{enumerate}

ซึ่งจากกราฟ \ref{fig:time_slot} 
\begin{figure}
  \begin{center}
    \includegraphics[width=\linewidth]{images/bar_chart_for_final_exam_slot.png}
  \end{center}
  \caption[Poem]{ความต้องการในสอบของผู้ตอบแบบสอบถามในแต่ละช่วงเวลา}
  \label{fig:time_slot}     
\end{figure}
ยังสามารถสรุปได้ว่าผู้ทำแบบสอบถามส่วนใหญ่ต้องการที่จะสอบหนึ่งวันเว้นหนึ่งวันเพื่อที่จะได้มีเวลาในการอ่านหนังสือเตรียมสอบสำหรับวิชาในวันถัดไปมากกว่าการสอบติดกัน 
เรายังสามารถบอกเพิ่มเติมได้อีกว่าผู้ทำแบบสอบถามส่วนใหญ่ต้องการสอบในช่วงสัปดาห์แรกของช่วงการสอบมากกว่าช่วงสัปดาห์ที่สองเพื่อที่จะได้มีเวลาพักผ่อนหรือกลับบ้าน หลังจากที่สอบเสร็จแล้ว

\section{โครงสร้างและการทำงานของโปรแกรม}
\subsection{ข้อมูลที่จำเป็นสำหรับการพัฒนาโปรแกรม}
\begin{enumerate}
  \item  ข้อมูล (Temp)
\end{enumerate}
\subsection{Input}
สำหรับ Input ของโปรแกรมจัดตารางสอบนั้นจะเป็นไฟล์ text จำนวน x ไฟล์ โดยไฟล์ที่หนึ่งประกอบไปด้วย ... และ ... 
ไฟล์ที่สอง ประกอบไปด้วย ... และ ...
ไฟล์ที่ n
\subsection{Output}
Output ของโปรแกรมจะมี format เป็นไฟล์ text ที่ระบุรายวิชาคู่กับวันและเวลาที่สอบ




