\documentclass[semifinal]{cpecmu}

% This is a sample document demonstrating how to use the CPECMU
% project template. If you are having trouble, see "cpecmu.pdf" for
% documentation.

\projectNo{69}
\acadyear{2020}

\titleTH{โครงงานสุดเลิฟของฉัน}
\titleEN{Your Project Name Goes Here}

\author{นายกินรี ไทร์ล้ำเลิศ}{Kinnaree Tirelumlert}{690610696}
\author{นายบรรจบ พบเอฟตลอด}{Banjob Pob-eftalord}{690610969}

\cpeadvisor{chinawat}
\cpecommittee{paskorn}
\committee{รศ.ดร.\,นิพนธ์ ธีรอำพน}{Assoc.\,Prof.\,Nipon Theera-Umpon, Ph.D.}

% Some possible packages to include:
\usepackage[final]{graphicx} % for including graphics

% Add bookmarks and hyperlinks in the document.
\usepackage[colorlinks=true,allcolors=Blue4,citecolor=red,linktoc=all]{hyperref}

% Needed just by this example, but maybe not by most reports
\usepackage{afterpage} % for outputting
\usepackage{pdflscape} % for landscape figures and tables. 

% Some other useful packages. Look these up to find out how to use
% them.
% \usepackage{natbib}    % for author-year citation styles
% \usepackage{txfonts}
% \usepackage{appendix}  % for appendices on a per-chapter basis
% \usepackage{xtab}      % for tables that go over multiple pages
% \usepackage{subfigure} % for subfigures within a figure
% \usepackage{pstricks,pdftricks} % for access to special PostScript and PDF commands
% \usepackage{nomencl}   % if you have a list of abbreviations

% if you're having problems with overfull boxes, you may need to increase
% the tolerance to 9999
% \tolerance=9999

\bibliographystyle{plain}
% \bibliographystyle{IEEEbib}

% \renewcommand{\topfraction}{0.85}
% \renewcommand{\textfraction}{0.1}
% \renewcommand{\floatpagefraction}{0.75}

% Example for glossary entry
% Need to use glossary option
% See glossaries package for complete documentation.
\ifglossary
  \newglossaryentry{lorem ipsum}{
    name=lorem ipsum,
    description={derived from Latin dolorem ipsum, translated as ``pain itself''}
  }
\fi

\begin{document}
\maketitle
\makesignature

\ifproject
\begin{abstractTH}
เขียนบทคัดย่อของโครงงานที่นี่

การเขียนรายงานเป็นส่วนหนึ่งของการทำโครงงานวิศวกรรมคอมพิวเตอร์
เพื่อทบทวนทฤษฎีที่เกี่ยวข้อง อธิบายขั้นตอนวิธีแก้ปัญหาเชิงวิศวกรรม และวิเคราะห์และสรุปผลการทดลองอุปกรณ์และระบบต่างๆ
\enskip อย่างไรก็ดี การสร้างรูปเล่มรายงานให้ถูกรูปแบบนั้นเป็นขั้นตอนที่ยุ่งยาก
แม้ว่าจะมีต้นแบบสำหรับใช้ในโปรแกรม Microsoft Word แล้วก็ตาม
แต่นักศึกษาส่วนใหญ่ยังคงค้นพบว่าการใช้งานมีความซับซ้อน และเกิดความผิดพลาดในการจัดรูปแบบ กำหนดเลขหัวข้อ และสร้างสารบัญอยู่
\enskip ภาควิชาวิศวกรรมคอมพิวเตอร์จึงได้จัดทำต้นแบบรูปเล่มรายงานโดยใช้ระบบจัดเตรียมเอกสาร
\LaTeX{} เพื่อช่วยให้นักศึกษาเขียนรายงานได้อย่างสะดวกและรวดเร็วมากยิ่งขึ้น
\end{abstractTH}

\begin{abstract}
The abstract would be placed here. It usually does not exceed 350 words
long (not counting the heading), and must not take up more than one (1) page
(even if fewer than 350 words long).

Make sure your abstract sits inside the \texttt{abstract} environment.
\end{abstract}

\iffalse
\begin{dedication}
This document is dedicated to all Chiang Mai University students.

Dedication page is optional.
\end{dedication}
\fi % \iffalse

\begin{acknowledgments}
Your acknowledgments go here. Make sure it sits inside the
\texttt{acknowledgment} environment.

\acksign{2020}{5}{25}
\end{acknowledgments}%
\fi % \ifproject

\contentspage

\ifproject
\figurelistpage

\tablelistpage
\fi % \ifproject

% \abbrlist % this page is optional

% \symlist % this page is optional

% \preface % this section is optional

\pagestyle{empty}\cleardoublepage
\normalspacing \setcounter{page}{1} \pagenumbering{arabic} \pagestyle{cpecmu}

\chapter{\ifcpe บทนำ\else Introduction\fi}

\section{\ifcpe ที่มาของโครงงาน\else Project rationale\fi}

\section{\ifcpe วัตถุประสงค์ของโครงงาน\else Objectives\fi}
\begin{enumerate}
    \item
\end{enumerate}

\section{\ifcpe ขอบเขตของโครงงาน\else Project scope\fi}

\subsection{\ifcpe ขอบเขตด้านฮาร์ดแวร์\else Hardware scope\fi}

\subsection{\ifcpe ขอบเขตด้านซอฟต์แวร์\else Software scope\fi}

\section{\ifcpe ประโยชน์ที่ได้รับ\else Expected outcomes\fi}

\section{\ifcpe เทคโนโลยีและเครื่องมือที่ใช้\else Technology and tools\fi}

\subsection{\ifcpe เทคโนโลยีด้านฮาร์ดแวร์\else Hardware technology\fi}

\subsection{\ifcpe เทคโนโลยีด้านซอฟต์แวร์\else Software technology\fi}

\section{\ifcpe แผนการดำเนินงาน\else Project plan\fi}

\begin{plan}{6}{2020}{2}{2021}
    \planitem{7}{2020}{8}{2020}{ศึกษาค้นคว้า}
    \planitem{8}{2020}{1}{2021}{ชิล}
    \planitem{2}{2021}{2}{2021}{เผา}
    \planitem{12}{2019}{1}{2022}{ทดสอบ}
\end{plan}

\section{\ifcpe บทบาทและความรับผิดชอบ\else Roles and responsibilities\fi}
อธิบายว่าในการทำงาน นศ. มีการกำหนดบทบาทและแบ่งหน้าที่งานอย่างไรในการทำงาน จำเป็นต้องใช้ความรู้ใดในการทำงานบ้าง

\section{\ifcpe%
ผลกระทบด้านสังคม สุขภาพ ความปลอดภัย กฎหมาย และวัฒนธรรม
\else%
Impacts of this project on society, health, safety, legal, and cultural issues
\fi}

แนวทางและโยชน์ในการประยุกต์ใช้งานโครงงานกับงานในด้านอื่นๆ รวมถึงผลกระทบในด้านสังคมและสิ่งแวดล้อมจากการใช้ความรู้ทางวิศวกรรมที่ได้

\chapter{\ifcpe ทฤษฎีที่เกี่ยวข้อง\else Background Knowledge and Theory\fi}

การทำโครงงานนี้เริ่มต้นจากการที่เราเล็งเห็นปัญหาของตารางสอบปลายภาคของมหาวิทยาลัยเชียงใหม่ 
ซึ่งตารางสอบของนักศึกษาบางคนอาจจะมีตารางเวลาที่ติดกันมากเกินไป ซึ่งผู้จัดทำเห็นว่าปัญหาการจัดตารางสอบปลายภาค
สามารถแปลงเป็นปัญหาที่มีวิธีในการแก้ไขอยู่แล้วได้ ในบทนี้จะกล่าวถึงผลการศึกษาค้นคว้าทฤษฎีที่เกี่ยวข้อง งานวิจัย หรือโครงงาน ที่เคยมีผู้นำเสนอไว้แล้ว
เพื่อช่วยอธิบายถึงสิ่งต่าง ๆ ที่เกี่ยวข้องกับโครงงานนี้เพื่อให้ผู้อ่านเข้าใจเนื้อหาในบทถัด ๆ ไปได้ง่ายยิ่งขึ้น โดยในบทนี้จะมีเนื้อหาต่าง ๆ ได้แก่ Literature review 
ซึ่งจะกล่าวถึงงานวิจัยต่าง ๆ ที่ได้ศึกษามา และส่วนของอัลกอลิทึมที่เกี่ยวข้อง ซึ่งจะกล่าวถึงรูปแบบและวิธีการทำงานของอัลกอลิทึมต่าง ๆ 
\section{Literature review}
การกำหนดเวลาสอบปลายภาคเพื่อหลีกเลี่ยงปัญหานักศึกษาคนใด ๆ มีเวลาสอบในช่วงเวลาเดียวกันสามารถแปลงปัญหานี้ให้เป็นปัญหา graph coloring~\cite{mcs} ได้ 
โดยที่จุดแต่ละจุดในกราฟเป็นรายวิชาที่เปิดสอนในภาคการศึกษานั้น 
และเส้นที่เชื่อมแต่ละจุดสองจุดในกราฟแสดงถึงการมีนักศึกษาที่ลงทะเบียนเรียนทั้งสองรายวิชา โดยจุดสองจุดใด ๆ ในกราฟที่มีเส้นเชื่อมกันจะถูกกำหนดสีให้ต่างกัน
ซึ่งแสดงถึงวันและเวลาที่จัดสอบปลายภาคในวิชานั้น ๆ โดยจุดที่มีคนละสีก็จะถูกจัดให้สอบคนละช่วงเวลากัน

การแปลงปัญหาการจัดตารางสอบปลายภาคให้เป็นปัญหา graph coloring จะสามารถแก้ปัญหาการจัดตารางสอบแบบพื้นฐานได้เท่านั้น โดยไม่คำนึงถึงข้อจำกัดอย่างอื่น 
ตัวอย่างเช่น ไม่พิจารณาความจุที่นั่งสำหรับการสอบแต่ละช่วงเวลาของนักศึกษา จำนวนอาจารย์ที่คุมสอบแต่ละช่วงเวลา และการกระจายวิชาสอบสำหรับนักศึกษาแต่ละคน เป็นต้น
ถึงแม้จะไม่กำหนดข้อจำกัดใด ๆ ปัญหา graph coloring ก็เป็นปัญหา NP-complete ด้วยตัวมันเองอยู่แล้ว~\cite{alg-design} 
ซึ่งหมายความว่ายังไม่สามารถหาอัลกอริทึมที่ใช้เวลา polynomial-time ในการแก้ไขปัญหาให้ได้ผลลัพธ์ที่เหมาะสมที่สุด 
ทำให้ต้องใช้วิธีอื่นที่ให้ผลลัพธ์ที่ดีในระดับที่ยอมรับได้ แต่สามารถยืนยันได้ว่าจะได้วิธีการที่สามารถแก้ไขปัญหาได้อย่างแน่นอน 
ซึ่งวิธีการนั้นคือ metaheuristic ซึ่งสามารถหาวิธีการแก้ปัญหาที่ดีได้ในระยะเวลาที่เหมาะสม~\cite{meta-for-vertexcolor}
และสามารถกำหนดข้อจำกัดหรือเงื่อนไขอื่นเพิ่มเติมได้ ทำให้สามารถกำหนดขอบเขตของผลลัพธ์ได้ แต่อาจจะไม่ได้วิธีแก้ปัญหาที่ดีที่สุด

อีกวิธีการที่สามารถใช้แก้ไขปัญหาการจัดตารางสอบได้ก็คือการใช้ memetic algorithms (MA) ซึ่งเป็นวิธีการที่นำ local search มาประยุกต์ใช้กับ genetic algorithm 
เพื่อช่วยลดระยะเวลาให้คำตอบของปัญหานั้น converge ช้าลง \cite{pablo-memetic-algo} ซึ่ง Ender {\"O}zcan เคยได้นำวิธีการนี้มาประยุกต์ใช้ในการแก้ปัญหาการจัดตารางสอบ 
โดยได้สร้าง Framework สำหรับออกแบบตัวดำเนินการที่ใช้ในการ crossover และ mutation ของ genetic algorithm ด้วย~\cite{fes}
โดยในการทดลองนี้ได้มีการคำนึงถึงนักศึกษา โดยกำหนดข้อจำกัดของตารางสอบที่ได้ให้ไม่มีนักศึกษาที่ต้องสอบติดกันสองวิชาในแต่ละวัน แต่การจัดตารางสอบในแบบของ Ender 
นั้นเป็นการจัดตารางสอบของมหาวิทยาลัย Yeditepe โดยแบ่งตารางสอบเป็นของภาควิชาต่าง ๆ และแบ่งย่อยแยกตามสาขาวิชาอีกที วิธีการนี้ไม่สามารถนำมาใช้กับการจัดตารางสอบของมหาวิทยาลัยเชียงใหม่ได้
เนื่องจากมหาวิทยาลัยเชียงใหม่ มีวิชาศึกษาทั่วไปซึ่งเป็นวิชาที่เปิดให้นักศึกษาจากต่างคณะสามารถลงทะเบียนได้ทำให้มีนักศึกษาเป็นจำนวนมากกว่าที่ความจุที่นั่งสอบของคณะนั้นจะรับไหว
ซึ่งทำให้การจัดตารางสอบโดยใช้วิธีนี้นั้นเป็นไปได้ยากเพราะจำนวนนักศึกษาที่เกินความจุที่นั่งสอบนั้นจะละเมิดข้อจำกัดที่กำหนดไว้ 

\iffalse
\section{Tools}
\subsection{Gurobi Optimizer}
Gurobi Optimizer เป็น Solver ที่ใช้สำหรับแก้ปัญหา optimization โดยที่จะเน้นไปทางด้านของปัญหาต่าง ๆ ดังนี้ 
\begin{itemize}
  \item Linear programming (LP)
  \item Mixed-integer linear programming (MILP)
  \item Quadratic programming (QP)
  \item Mixed-integer quadratic programming (MIQP)
  \item Quadratically-constrained programming (QCP)
  \item Mixed-integer quadratically-constrained programming (MIQCP)
\end{itemize}
ผลลัพธ์ที่ได้จาก Gurobi Optimizer อาจนำมาใช้เป็นตัวเปรียบเทียบประสิทธิภาพกับผลลัพธ์การทำงานที่ได้จากอัลกอลิทึมของเรา
\fi
\section{Algorithm}
\subsection{Metaheuristic}
Meta­heuristic เป็นวิธีการแก้ไขปัญหาที่ใช้งานกันโดยทั่วไป ซึ่งเป็นวิธีการที่สามารถหาผลลัพธ์ของ optimization problems ได้ดีและเหมาะสมกับเวลาที่ใช้ในการประมวลผล~\cite{metaheuris}
โดยวิธีเหล่านี้นั้นส่วนใหญ่จะเป็นวิธีการที่มีแรงบันดาลใจมาจากการเรียนแบบหลักการของธรรมชาติและนำมาดัดแปลงเป็นอัลกอริทึมเพื่อใช้ในการแก้ปัญหาต่าง ๆ ตัวอย่างของ metaheuristic algorithms เช่น genetic algorithm, evolutionary computation, simulated annealing, tabu search เป็นต้น
\subsection{Genetic algorithms}
Genetic algorithm เป็นหนึ่งใน metaheuristic ซึ่งเป็นเทคนิคสำหรับค้นหาผลลัพธ์หรือคำตอบโดยประมาณของปัญหา โดยอาศัยหลักการจากทฤษฎีวิวัฒนาการทางชีววิทยาและหลักการคัดเลือกตามธรรมชาติ 
\linebreak กล่าวคือ สิ่งมีชีวิตที่เหมาะสมที่สุดจึงจะอยู่รอด โดยกระบวนการคัดเลือกได้เปลี่ยนแปลงสิ่งมีชีวิตให้เหมาะสมยิ่งขึ้นด้วยตัวดำเนินการทางพันธุกรรม เช่น การสืบพันธุ์ การแลกเปลี่ยนยีน การกลายพันธุ์ เป็นต้น โดยขั้นตอนการทำงานของ genetic algorithm มีดังนี้ 
\begin{enumerate}
  \item initial population เป็นขั้นตอนเริ่มต้นของอัลกอริทึมซึ่งจะทำการกำหนดชุดข้อมูลผลลัพธ์ เรียกชุดข้อมูลนี้ว่า population ซึ่งชุดข้อมูลนี้จะประกอบไปด้วยผลลัพธ์ที่ถูกเข้ารหัสในรูปแบบของสายอักขระที่แต่ละอักขระเป็นบิต 0 หรือ 1 เรียกแต่ละบิตนี้ว่า gene
  โดยที่แต่ละสายอักขระที่ประกอบจาก genes นี้เรียกว่า chromosome โดยชุดข้อมูลนี้อาจจะสุ่มแต่ละ gene ขึ้นมาเป็นค่าเริ่มต้น
  \item fitness function เป็น function สำหรับใช้ในการคัดเลือกชุดข้อมูลที่เหมาะสมให้สามารถอยู่ต่อไปได้ โดยจะมีการคำนวนค่า fitness scores ให้กับแต่ละ chromosome
  โดย fitness scores จะขึ้นอยู่กับความพอใจในผลลัพธ์ที่ได้ของผู้พัฒนา
  \item genetic operator คือวิธีการในการปรับเปลี่ยนรูปแบบโครงสร้างของ chromosome ที่เหมาะสมสำหรับรุ่นถัดไปของกระบวนการ ซึ่งมีวิธีการอยู่ 3 แบบหลัก ๆ ได้แก่
  \begin{itemize}
  \item selection เป็นการเลือกคู่ chromosome ของข้อมูลที่เหมาะสม เพื่อให้ chromosome คู่นี้ส่งต่อ gene ที่ดีแล้วไปยัง chromosome รุ่นถัดไป โดยจะเลือก chromosome ที่มีค่า fitness scores มากที่สุด
  \item crossover เป็นการสุ่มเลือกตำแหน่งระหว่าง genes จาก parent chromosome 1 คู่ โดย genes ด้านซ้ายหรือขวาของจุดแบ่งจะถูกสับเปลี่ยนกันระหว่าง parent chromosome คู่นั้น
  \item mutation เป็นการสุ่มกลับค่าของ gene ใน chromosome ให้มีค่าตรงกันข้ามโดยมีค่าความน่าจะเป็นต่ำ ๆ เพื่อป้องกันไม่ให้ผลลัพธ์ converge ก่อนที่ควรจะเป็น
\end{itemize}
\end{enumerate}
genetic algorithm สามารถจบการทำงานได้หลายวิธี โดยมีเงื่อนไขในการจบการทำงานดังนี้
\begin{itemize}
  \item จบการทำงานเมื่อ population เปลี่ยนผ่านไปถึงรุ่นที่ต้องการแล้ว 
  \item จบการทำงานหาก population ไม่มีการพัฒนาแล้ว หรือไม่มีการเปลี่ยนแปลงไปในทางที่ดีขึ้นเป็นระยะเวลาหนึ่ง
  \item จบการทำงานเมื่อ fitness scores ของ population มีค่าเท่าที่ต้องการแล้ว
\end{itemize}
\section{local search}
local search เป็นวิธีที่ใช่สำหรับค้นหาคำตอบรูปแบบหนึ่ง 


\section{\ifcpe%
ความรู้ตามหลักสูตรซึ่งถูกนำมาใช้หรือบูรณาการในโครงงาน
\else%
ISNE knowledge used, applied, or integrated in this project
\fi
}
\begin{itemize}
  \item 261218 algorithm for computer engineering ได้นำวิธีดำเนินงาน หลักการและทฤษฏี ดังนี้มาใช้เพื่อแก้ไขปัญหาในโครงงานนี้  
  \begin{itemize}
  \item วิธีการคิดและวิเคราะห์ปัญหา
  \item วิธีการแปลงปัญหาใหญ่ที่แก้ไขยากให้กลายเป็นปัญหาที่เล็กกว่าเพื่อใช้วิธีการที่มีอยู่แล้วในการแก้ไขปัญหานั้น
  \item ทฤษฏี graph coloring
  \item การแก้ปัญหา NP-complete และ NP-Hard
  \end{itemize}
\end{itemize}

\section{\ifcpe%
ความรู้นอกหลักสูตรซึ่งถูกนำมาใช้หรือบูรณาการในโครงงาน
\else%
Extracurricular knowledge used, applied, or integrated in this project
\fi
}
ความรู้นอกหลักสูตรที่ใช้สำหรับการแก้ไขปัญหาของโครงงานเพื่อให้ได้ผลลัพธ์ที่เหมาะสุดที่สุด เราได้ทำการศึกษา หลักการและทฤษฏี ดังนี้
\begin{itemize}
  \item Meta­heuristic 
  \item Genetic algorithm
  \item Local search
\end{itemize}

\chapter{\ifproject%
\ifcpe โครงสร้างและขั้นตอนการทำงาน\else Project Structure and Methodology\fi
\else%
\ifcpe โครงสร้างของโครงงาน\else Project Structure\fi
\fi
}

ในบทนี้จะกล่าวถึงหลักการ และการออกแบบระบบ

\makeatletter

% \renewcommand\section{\@startsection {section}{1}{\z@}%
%                                    {13.5ex \@plus -1ex \@minus -.2ex}%
%                                    {2.3ex \@plus.2ex}%
%                                    {\normalfont\large\bfseries}}

\makeatother
%\vspace{2ex}
% \titleformat{\section}{\normalfont\bfseries}{\thesection}{1em}{}
% \titlespacing*{\section}{0pt}{10ex}{0pt}

\section{ผลการสำรวจของนักศึกเกี่ยวกับตารางสอบปลายภาค}
จากการทำแบบสอบถามเกี่ยวกับตารางสอบปลายภาคของมหาวิทยาลัยชียงใหม่ โดยขอความร่วมมือนักศึกษาทุกระดับชั้นในมหาวิทยาลัยเชียงใหม่ 
ทั้งนักศึกษาที่กำลังศึกษาอยู่รวมทั้งที่สำเร็จการศึกษาไปแล้ว เพื่อสอบถามความคิดเห็นเกี่ยวกับ ข้อดี ข้อเสีย ของตารางสอบที่แต่ละคนได้รับ 
ร่วมถึงปัญหาที่เคยกระทบมา โดยเราสามารถบอกได้ว่าผู้ทำแบบสอบถามส่วนใหญ่นั้นตรวจสอบดูช่วงเวลาสอบของแต่ละวิชาก่อนที่จะลงทะเบียนอย่างสม่ำเสมอ 
แต่ยังมีผู้ทำแบบสอบถามบางส่วนนั้นตรวจสอบดูช่วงเวลาสอบก่อนจะลงทะเบียนเรียนในบางวิชาไม่สม่ำ และมีผู้ทำแบบสอบถามส่วนน้อยที่ไม่มีเคยตรวจสอบปลายภาคอยู่เลยดังรูปที่ \ref{fig:enroll}
ต่อมาเราได้สอบถามเกี่ยวกับการจัดตารางสอบของสำนักทะเบียนมหาลัยเชียงใหม่ได้ข้อผลสรุปว่าผู้ทำแบบสอบถามส่วนใหญ่นั้นไม่ทราบถึงเวลาการจัดตารางสอบของสำนักทะเบียน มีคนที่ทราบเป็นส่วนน้อยเท่าน้ันที่ทราบถึงวิธีการจัดตารางสอบดังรูปที่ \ref{fig:Create_exam} 
่ต่อมาเราได้สอบถามเวลาที่ผู้ทำแบบสอบถามต้องการจะสอบคือช่วงเวลาใดของสัปดาห์โดยได้ผลลัพธ์ดังรูป \ref{fig:time_slot} โดย 7 ลำดับช่วงเวลาที่ผู้ทำแบบสอบถามอยากสอบมากที่สุดเป็นดังนี้(จากมากไปน้อยตามลำดับ)
\begin{enumerate}
  \item สัปดาห์ที่หนึ่ง เวลา 12.00-15.00น. วันศุกร์ 
  \item สัปดาห์ที่หนึ่ง เวลา 12.00-15.00น. วันจันทร์
  \item สัปดาห์ที่หนึ่ง เวลา 12.00-15.00น. วันพุธ
  \item สัปดาห์ที่สอง เวลา 12.00-15.00น. วันอังคาร
  \item สัปดาห์ที่หนึ่ง เวลา 12.00-15.00น. วันอาทิตย์
  \item สัปดาห์ที่สอง เวลา 12.00-15.00น. วันพฤหัสบดี
  \item สัปดาห์ที่สอง เวลา 12.00-15.00น. วันเสาร์
\end{enumerate}

จากข้อมูลเราสามารถบอกได้ว่าผู้ทำแบบสอบถามส่วนใหญ่ต้องการที่จะสอบในแต่ละวันคือช่วงเวลา 12.00-15.00น. ซึ่งผู้ทำแบบสอบถามส่วนมากต้องการจะสอบช่วงเวลานี้มากกว่า 15.30-18.00น. และ 08.00-11.00น. ตามลำดับ ดังรูป \ref{fig:time} 
ุ้และถ้าเราทำการแยกช่วงเวลาที่ผู้ตอบแบบสอบถามต้องการสอบออกเป็นวัน เรายังสามารถบอกได้ว่าผู้ทำแบบสอบถามส่วนใหญ่ต้องการจะสอบต้องการที่จะสอบหนึ่งวันเว้นหนึ่งวันเพื่อที่จะได้มีเวลาในการอ่านหนังสือเตรียมสอบสำหรับวิชาในวันต่อไปมากกว่าสอบติดกัน 
ยังสามารถบอกเพิ่มเติมได้อีกว่าผู้ทำแบบสอบถามส่วนใหญ่ต้องการสอบในช่วงสัปดาห์แรกของช่วงการสอบมากกว่าช่วงสัปดาห์สุดของการสอบเพื่อที่จะมีเวลาพักผ่อนหลังจากทำการสอบทั้งหมดดังรูปที่ \ref{fig:day}

\begin{figure}
\begin{center}
\includegraphics{images/check_enrollment.png}\\[2ex]
\includegraphics{images/type_check_enrollment.png}
\end{center}
\caption[Poem]{จำนวนผู้ตอบแบบสอบถามที่ตรวจสอบตารางสอบปลายภาคก่อนการลงทะเบียน}
\label{fig:enroll}     
\end{figure}

\begin{figure}
\begin{center}
\includegraphics{images/Create_exam.png}\\[2ex]
\includegraphics{images/type_Create_exam.png}
\end{center}
\caption[Poem]{จำนวนผู้ตอบแบบสอบถามที่ทราบวิธีการจัดตารางสอบปลายภาคของสำนักทะเบียน}
\label{fig:Create_exam}     
\end{figure}

\begin{figure}
\begin{center}
\includegraphics[width=\linewidth]{images/bar_chart.png}
\end{center}
\caption[Poem]{ความต้องการในสอบของผู้ตอบแบบสอบถามในแต่ละช่วงเวลา}
\label{fig:time_slot}     
\end{figure}

\begin{figure}
\begin{center}
\includegraphics[width=\linewidth]{images/chart.png}
\end{center}
\caption[Poem]{ความต้องการในสอบของผู้ตอบแบบสอบถามในแต่ละวัน}
\label{fig:day}     
\end{figure}

\begin{figure}
\begin{center}
\includegraphics[width=\linewidth]{images/pie.png}
\end{center}
\caption[Poem]{ความต้องการในสอบของผู้ตอบแบบสอบถามในแต่ละเวลา}
\label{fig:time}     
\end{figure}



\chapter{\ifproject%
\ifcpe การทดลองและผลลัพธ์\else Experimentation and Results\fi
\else%
\ifcpe การประเมินระบบ\else System Evaluation\fi
\fi}

ในบทนี้จะทดสอบเกี่ยวกับการทำงานในฟังก์ชันหลักๆ

\section{การประเมินระบบโดยการวัดค่า Penalty}
ในการประเมินผลระบบจัดตารางสอบที่จะพัฒนาขึ้นมานั้นจะมีการประเมินความสมดุลและความเหมาะสมของตารางสอบที่เป็นผลลัพธ์ของระบบ จากค่าเฉลี่ยของ Penalty 
โดยค่า Penalty นั้นจะคำนวณมาจากตารางสอบของนักศึกษาแต่ละคน
โดยจะมีการนำค่าเฉลี่ย Penalty มาเปรียบเทียบกันระหว่างตารางสอบทั้ง 2 แบบ
ได้แก่
\begin{itemize}
    \item ตารางสอบที่เป็นผลลัพธ์จากระบบ
    \item ตารางสอบดั้งเดิมจากสำนักทะเบียน
\end{itemize}

\subsection{การคิด Penalty}

\section{การประเมินระบบโดยสอบถามความพึงพอใจของนักศึกษา}
\ifproject
\chapter{\ifcpe บทสรุปและข้อเสนอแนะ\else Conclusions and Discussion\fi}

\section{\ifcpe สรุปผล\else Conclusions\fi}

นศ. ควรสรุปถึงข้อจำกัดของระบบในด้านต่างๆ ที่ระบบมีในเนื้อหาส่วนนี้ด้วย

\section{\ifcpe ปัญหาที่พบและแนวทางการแก้ไข\else Challenges\fi}

ในการทำโครงงานนี้ พบว่าเกิดปัญหาหลักๆ ดังนี้

\section{\ifcpe%
ข้อเสนอแนะและแนวทางการพัฒนาต่อ
\else%
Suggestions and further improvements
\fi
}

ข้อเสนอแนะเพื่อพัฒนาโครงงานนี้ต่อไป มีดังนี้

\fi

\bibliography{sampleReport}

\ifproject
\appendix
\chapter{The first appendix}

Text for the first appendix goes here.

\section{Appendix section}

Text for a section in the first appendix goes here.

\chapter{\ifcpe คู่มือการใช้งานระบบ\else Manual\fi}

Manual goes here.

% Display glossary (optional) -- need glossary option.
\ifglossary\glossarypage\fi

% Display index (optional) -- need idx option.
\ifindex\indexpage\fi

\begin{biosketch}
\begin{center}
  \includegraphics[width=1.5in]{mugshot.jpg}
\end{center}
Your biosketch goes here. Make sure it sits inside
the \texttt{biosketch} environment.
\end{biosketch}
\fi % \ifproject
\end{document}
