\documentclass[semifinal]{cpecmu}

\projectNo{P001-2}
\acadyear{2020}

\titleTH{ระบบจัดตารางสอบปลายภาค}
\titleEN{Final exam scheduling system}

\author{นายกฤษฏิ์ อุปนันท์}{Krit Upanun}{600610717}
\author{นายธนวงศ์ เสนีวงศ์ ณ อยุธยา}{Thanawong Saneewong Na Ayutthaya}{600610738}

\cpeadvisor{chinawat}
\cpecommittee{pruet}
\cpecommittee{sanpawat}%

%% Some possible packages to include:
\usepackage[final]{graphicx} % for including graphics

%% Add bookmarks and hyperlinks in the document.
\PassOptionsToPackage{hyphens}{url}
\usepackage[colorlinks=true,allcolors=Blue4,citecolor=red,linktoc=all]{hyperref}

%% Set up commenting
\iffinal
  \usepackage[disabled]{authcomments}
\else
  \usepackage{authcomments}
\fi
\newcommenter{CI}{0.0,0.5625,0.0}  % green
\newcommenter{TSNA}{0.0,0.0,0.7}
\newcommenter{KT}{0.7,0.0,0.0}
\usepackage{enumitem}

%% Needed just by this example, but maybe not by most reports
\usepackage{afterpage} % for outputting
\usepackage{pdflscape} % for landscape figures and tables. 

\usepackage{forest}
\definecolor{folderbg}{RGB}{124,166,198}
\definecolor{folderborder}{RGB}{110,144,169}
\def\Size{4pt}
\tikzset{
      folder/.pic={
        \filldraw[draw=folderborder,top color=folderbg!50,bottom color=folderbg]
          (-1.05*\Size,0.2\Size+5pt) rectangle ++(.75*\Size,-0.2\Size-5pt);  
        \filldraw[draw=folderborder,top color=folderbg!50,bottom color=folderbg]
          (-1.15*\Size,-\Size) rectangle (1.15*\Size,\Size);
      }
}

\usepackage{multicol}
\usepackage{multirow}
\usepackage{booktabs}
\usepackage{graphicx}

\usepackage{tikz}
\usetikzlibrary{positioning}
\usetikzlibrary{arrows.meta}
\tikzset{>/.tip={Latex[length=2mm]}}
\tikzset{arrow/.style={->}}
\tikzset{double distance=2pt}
\tikzset{flowchart/.style = {node distance=0.75cm and 1cm,
     base/.style = {draw=black,
                    inner sep=1mm, outer sep=0mm,
                    minimum height=1cm,
                    align=flush center},
startstop/.style = {base, rounded corners=5mm, minimum width=1in},
  process/.style = {base, inner xsep=0.5cm},
 decision/.style = {base, diamond, aspect=3},
connector/.style = {draw, circle, minimum size=0pt},
       io/.style = {base, trapezium,
                    trapezium left angle=60, trapezium right angle=-60,
                    trapezium stretches=true},
        io2/.style = {base, trapezium,
                    trapezium left angle=75, trapezium right angle=-75,
                    trapezium stretches=true}
                    }}
%% Some other useful packages. Look these up to find out how to use
%% them.
% \usepackage{natbib}    % for author-year citation styles
% \usepackage{txfonts}
% \usepackage{appendix}  % for appendices on a per-chapter basis
% \usepackage{xtab}      % for tables that go over multiple pages
% \usepackage{subfigure} % for subfigures within a figure
% \usepackage{pstricks,pdftricks} % for access to special PostScript and PDF commands
% \usepackage{nomencl}   % if you have a list of abbreviations

%% if you're having problems with overfull boxes, you may need to increase
%% the tolerance to 9999
% \tolerance=9999

\bibliographystyle{plain}
% \bibliographystyle{IEEEbib}

% \renewcommand{\topfraction}{0.85}
% \renewcommand{\textfraction}{0.1}
% \renewcommand{\floatpagefraction}{0.75}

%% Example for glossary entry
%% Need to use glossary option
%% See glossaries package for complete documentation.
\ifglossary
  \newglossaryentry{lorem ipsum}{
    name=lorem ipsum,
    description={derived from Latin dolorem ipsum, translated as ``pain itself''}
  }
\fi

%% Uncomment this command to preview only specified LaTeX file(s)
%% imported with \include command below.
%% Any other file imported via \include but not specified here will not
%% be previewed.
%% Useful if your report is large, as you might not want to build
%% the entire file when editing a certain part of your report.
% \includeonly{chapters/intro,chapters/background}

\begin{document}
\maketitle
\makesignature

\ifproject
\begin{abstractTH}
    

ปัจจุบันตารางสอบของนักศึกษาทุกระดับในมหาวิทยาลัยเชียงใหม่ มีการจัดตารางสอบโดยแบ่งกระบวนวิชาออกเป็นสองส่วน 
ได้แก่ กระบวนวิชาปกติคือกระบวนวิชาที่มีตอนเดียว และกระบวนวิชาพิเศษคือกระบวนวิชาที่มีมากกว่าหนึ่งตอน
โดยกระบวนวิชาปกติจะถูกนำไปจัดตารางสอบโดยอ้างอิงตามช่วงเวลาที่เรียนของแต่ละกระบวนวิชา ซึ่งถูกกำหนดเอาไว้ชัดเจน
ส่วนของกระบวนวิชาพิเศษจะนำไปจัดตารางสอบโดยเลือกเวลาจัดสอบของแต่ละกระบวนวิชานั้นที่ไม่ตรงกับเวลาจัดสอบของกระบวนวิชาปกติและอยู่ในช่วงของสัปดาห์ที่จัดการสอบ
เพื่อแก้ปัญหาให้วิชาที่มีมากกว่าหนึ่งตอนนั้นสามารถสอบในเวลาเดียวกันได้
จากการจัดตารางสอบด้วยวิธีข้างต้นนี้ทำให้ตารางสอบของนักศึกษาบางคนเกิดปัญหาขึ้น
ซึ่งทำให้นักศึกษาไม่สามารถเลือกลงทะเบียนกระบวนวิชาที่สนใจได้อย่างอิสระ เนื่องจากตารางสอบที่ซับซ้อนกัน
เวลาสอบติดกันตั้งแต่สองวิชาขึ้นไปในวันเดียว อีกทั้งบางกระบวนวิชาปกตินั้นยังมีการกำหนดวันสอบอยู่ในช่วงวันสุดท้ายของสัปดาห์ที่จัดสอบทำให้นักศึกษาบางคนว่าง
ระหว่างกลางสัปดาห์สอบเป็นเวลาหลายวัน จากปัญหาที่ได้กล่าวมานั้น ทางทีมผู้จัดทำจึงได้พัฒนาระบบจัดตารางสอบขึ้นมา 
โดยใช้ข้อมูลจากการลงทะเบียนของนักศึกษา จำนวนกระบวนวิชาทั้งหมดที่เปิดสอน และคู่กระบวนวิชาที่มีนักศึกษาลงทะเบียนทั้งสองกระบวนวิชาพร้อมกัน 
เพื่อนำมาสร้างข้อจำกัดสำหรับนำมาใช้กับโปรแกรมแก้ปัญหา(Solver ?) เพื่อหาตารางสอบที่ดีที่สุดที่เป็นไปได้จากข้อจำกัดต่าง ๆ 
โดยสามารถประเมินว่าตารางสอบเป็นตารางสอบที่ดีได้หากตารางสอบที่ได้จากโปรแกรมแก้ปัญหามีค่า penalty ? น้อย 

\CI{}{เรียงลำดับความคิดดังนี้
\begin{itemize}[nosep,leftmargin=*]
    \item เกริ่นถึงวิธีการจัดตารางสอบในปัจจุบัน (เน้น regular exam ที่จัดตามตารางเรียน แล้วพูดถึง special ที่พยายามแก้ปัญหาวิชาที่มีหลาย sections)
    \item พูดถึงปัญหาที่เกิดขึ้น (สอบชนกัน สอบหลายวิชาในวันเดียวกัน สอบติดกันหลายวิชา ว่างสอบนานหลายวัน)
    \item จะ focus เฉพาะ มช หรือไม่ หรือใช้ได้กับที่อื่นด้วย?
    \item แนะนำระบบ ที่จะช่วยลดปัญหาดังกล่าว
    \item ระบบใช้ข้อมูลอะไรบ้างที่จะช่วยทำให้การสอบ balanced มากขึ้น
    \item ใช้วิธีอะไรในการจัดตารางสอบ (ปรับได้ทีหลัง)
    \item พูดถึงการทดสอบและประเมินระบบ? (ปรับได้ทีหลัง)
\end{itemize}
}

\end{abstractTH}

\begin{abstract}
The abstract would be placed here. It usually does not exceed 350 words
long (not counting the heading), and must not take up more than one (1) page
(even if fewer than 350 words long).

Make sure your abstract sits inside the \texttt{abstract} environment.
\end{abstract}

\iffalse
\begin{dedication}
This document is dedicated to all Chiang Mai University students.

Dedication page is optional.
\end{dedication}
\fi % \iffalse

\begin{acknowledgments}
Your acknowledgments go here. Make sure it sits inside the
\texttt{acknowledgment} environment.

\acksign{2020}{5}{25}
\end{acknowledgments}%
\fi % \ifproject

\contentspage

\ifproject
\figurelistpage

\tablelistpage
\fi % \ifproject

% \abbrlist % this page is optional

% \symlist % this page is optional

% \preface % this section is optional


\pagestyle{empty}\cleardoublepage
\normalspacing \setcounter{page}{1} \pagenumbering{arabic} \pagestyle{cpecmu}

\chapter{\ifcpe บทนำ\else Introduction\fi}

\section{\ifcpe ที่มาของโครงงาน\else Project rationale\fi}

\section{\ifcpe วัตถุประสงค์ของโครงงาน\else Objectives\fi}
\begin{enumerate}
    \item
\end{enumerate}

\section{\ifcpe ขอบเขตของโครงงาน\else Project scope\fi}

\subsection{\ifcpe ขอบเขตด้านฮาร์ดแวร์\else Hardware scope\fi}

\subsection{\ifcpe ขอบเขตด้านซอฟต์แวร์\else Software scope\fi}

\section{\ifcpe ประโยชน์ที่ได้รับ\else Expected outcomes\fi}

\section{\ifcpe เทคโนโลยีและเครื่องมือที่ใช้\else Technology and tools\fi}

\subsection{\ifcpe เทคโนโลยีด้านฮาร์ดแวร์\else Hardware technology\fi}

\subsection{\ifcpe เทคโนโลยีด้านซอฟต์แวร์\else Software technology\fi}

\section{\ifcpe แผนการดำเนินงาน\else Project plan\fi}

\begin{plan}{6}{2020}{2}{2021}
    \planitem{7}{2020}{8}{2020}{ศึกษาค้นคว้า}
    \planitem{8}{2020}{1}{2021}{ชิล}
    \planitem{2}{2021}{2}{2021}{เผา}
    \planitem{12}{2019}{1}{2022}{ทดสอบ}
\end{plan}

\section{\ifcpe บทบาทและความรับผิดชอบ\else Roles and responsibilities\fi}
อธิบายว่าในการทำงาน นศ. มีการกำหนดบทบาทและแบ่งหน้าที่งานอย่างไรในการทำงาน จำเป็นต้องใช้ความรู้ใดในการทำงานบ้าง

\section{\ifcpe%
ผลกระทบด้านสังคม สุขภาพ ความปลอดภัย กฎหมาย และวัฒนธรรม
\else%
Impacts of this project on society, health, safety, legal, and cultural issues
\fi}

แนวทางและโยชน์ในการประยุกต์ใช้งานโครงงานกับงานในด้านอื่นๆ รวมถึงผลกระทบในด้านสังคมและสิ่งแวดล้อมจากการใช้ความรู้ทางวิศวกรรมที่ได้

\chapter{\ifcpe ทฤษฎีที่เกี่ยวข้อง\else Background Knowledge and Theory\fi}

การทำโครงงานนี้เริ่มต้นจากการที่เราเล็งเห็นปัญหาของตารางสอบปลายภาคของมหาวิทยาลัยเชียงใหม่ 
ซึ่งตารางสอบของนักศึกษาบางคนอาจจะมีตารางเวลาที่ติดกันมากเกินไป ซึ่งผู้จัดทำเห็นว่าปัญหาการจัดตารางสอบปลายภาค
สามารถแปลงเป็นปัญหาที่มีวิธีในการแก้ไขอยู่แล้วได้ ในบทนี้จะกล่าวถึงผลการศึกษาค้นคว้าทฤษฎีที่เกี่ยวข้อง งานวิจัย หรือโครงงาน ที่เคยมีผู้นำเสนอไว้แล้ว
เพื่อช่วยอธิบายถึงสิ่งต่าง ๆ ที่เกี่ยวข้องกับโครงงานนี้เพื่อให้ผู้อ่านเข้าใจเนื้อหาในบทถัด ๆ ไปได้ง่ายยิ่งขึ้น โดยในบทนี้จะมีเนื้อหาต่าง ๆ ได้แก่ Literature review 
ซึ่งจะกล่าวถึงงานวิจัยต่าง ๆ ที่ได้ศึกษามา และส่วนของอัลกอลิทึมที่เกี่ยวข้อง ซึ่งจะกล่าวถึงรูปแบบและวิธีการทำงานของอัลกอลิทึมต่าง ๆ 
\section{Literature review}
การกำหนดเวลาสอบปลายภาคเพื่อหลีกเลี่ยงปัญหานักศึกษาคนใด ๆ มีเวลาสอบในช่วงเวลาเดียวกันสามารถแปลงปัญหานี้ให้เป็นปัญหา graph coloring~\cite{mcs} ได้ 
โดยที่จุดแต่ละจุดในกราฟเป็นรายวิชาที่เปิดสอนในภาคการศึกษานั้น 
และเส้นที่เชื่อมแต่ละจุดสองจุดในกราฟแสดงถึงการมีนักศึกษาที่ลงทะเบียนเรียนทั้งสองรายวิชา โดยจุดสองจุดใด ๆ ในกราฟที่มีเส้นเชื่อมกันจะถูกกำหนดสีให้ต่างกัน
ซึ่งแสดงถึงวันและเวลาที่จัดสอบปลายภาคในวิชานั้น ๆ โดยจุดที่มีคนละสีก็จะถูกจัดให้สอบคนละช่วงเวลากัน

การแปลงปัญหาการจัดตารางสอบปลายภาคให้เป็นปัญหา graph coloring จะสามารถแก้ปัญหาการจัดตารางสอบแบบพื้นฐานได้เท่านั้น โดยไม่คำนึงถึงข้อจำกัดอย่างอื่น 
ตัวอย่างเช่น ไม่พิจารณาความจุที่นั่งสำหรับการสอบแต่ละช่วงเวลาของนักศึกษา จำนวนอาจารย์ที่คุมสอบแต่ละช่วงเวลา และการกระจายวิชาสอบสำหรับนักศึกษาแต่ละคน เป็นต้น
ถึงแม้จะไม่กำหนดข้อจำกัดใด ๆ ปัญหา graph coloring ก็เป็นปัญหา NP-complete ด้วยตัวมันเองอยู่แล้ว~\cite{alg-design} 
ซึ่งหมายความว่ายังไม่สามารถหาอัลกอริทึมที่ใช้เวลา polynomial-time ในการแก้ไขปัญหาให้ได้ผลลัพธ์ที่เหมาะสมที่สุด 
ทำให้ต้องใช้วิธีอื่นที่ให้ผลลัพธ์ที่ดีในระดับที่ยอมรับได้ แต่สามารถยืนยันได้ว่าจะได้วิธีการที่สามารถแก้ไขปัญหาได้อย่างแน่นอน 
ซึ่งวิธีการนั้นคือ metaheuristic ซึ่งสามารถหาวิธีการแก้ปัญหาที่ดีได้ในระยะเวลาที่เหมาะสม~\cite{meta-for-vertexcolor}
และสามารถกำหนดข้อจำกัดหรือเงื่อนไขอื่นเพิ่มเติมได้ ทำให้สามารถกำหนดขอบเขตของผลลัพธ์ได้ แต่อาจจะไม่ได้วิธีแก้ปัญหาที่ดีที่สุด

อีกวิธีการที่สามารถใช้แก้ไขปัญหาการจัดตารางสอบได้ก็คือการใช้ memetic algorithms (MA) ซึ่งเป็นวิธีการที่นำ local search มาประยุกต์ใช้กับ genetic algorithm 
เพื่อช่วยลดระยะเวลาให้คำตอบของปัญหานั้น converge ช้าลง \cite{pablo-memetic-algo} ซึ่ง Ender {\"O}zcan เคยได้นำวิธีการนี้มาประยุกต์ใช้ในการแก้ปัญหาการจัดตารางสอบ 
โดยได้สร้าง Framework สำหรับออกแบบตัวดำเนินการที่ใช้ในการ crossover และ mutation ของ genetic algorithm ด้วย~\cite{fes}
โดยในการทดลองนี้ได้มีการคำนึงถึงนักศึกษา โดยกำหนดข้อจำกัดของตารางสอบที่ได้ให้ไม่มีนักศึกษาที่ต้องสอบติดกันสองวิชาในแต่ละวัน แต่การจัดตารางสอบในแบบของ Ender 
นั้นเป็นการจัดตารางสอบของมหาวิทยาลัย Yeditepe โดยแบ่งตารางสอบเป็นของภาควิชาต่าง ๆ และแบ่งย่อยแยกตามสาขาวิชาอีกที วิธีการนี้ไม่สามารถนำมาใช้กับการจัดตารางสอบของมหาวิทยาลัยเชียงใหม่ได้
เนื่องจากมหาวิทยาลัยเชียงใหม่ มีวิชาศึกษาทั่วไปซึ่งเป็นวิชาที่เปิดให้นักศึกษาจากต่างคณะสามารถลงทะเบียนได้ทำให้มีนักศึกษาเป็นจำนวนมากกว่าที่ความจุที่นั่งสอบของคณะนั้นจะรับไหว
ซึ่งทำให้การจัดตารางสอบโดยใช้วิธีนี้นั้นเป็นไปได้ยากเพราะจำนวนนักศึกษาที่เกินความจุที่นั่งสอบนั้นจะละเมิดข้อจำกัดที่กำหนดไว้ 

\iffalse
\section{Tools}
\subsection{Gurobi Optimizer}
Gurobi Optimizer เป็น Solver ที่ใช้สำหรับแก้ปัญหา optimization โดยที่จะเน้นไปทางด้านของปัญหาต่าง ๆ ดังนี้ 
\begin{itemize}
  \item Linear programming (LP)
  \item Mixed-integer linear programming (MILP)
  \item Quadratic programming (QP)
  \item Mixed-integer quadratic programming (MIQP)
  \item Quadratically-constrained programming (QCP)
  \item Mixed-integer quadratically-constrained programming (MIQCP)
\end{itemize}
ผลลัพธ์ที่ได้จาก Gurobi Optimizer อาจนำมาใช้เป็นตัวเปรียบเทียบประสิทธิภาพกับผลลัพธ์การทำงานที่ได้จากอัลกอลิทึมของเรา
\fi
\section{Algorithm}
\subsection{Metaheuristic}
Meta­heuristic เป็นวิธีการแก้ไขปัญหาที่ใช้งานกันโดยทั่วไป ซึ่งเป็นวิธีการที่สามารถหาผลลัพธ์ของ optimization problems ได้ดีและเหมาะสมกับเวลาที่ใช้ในการประมวลผล~\cite{metaheuris}
โดยวิธีเหล่านี้นั้นส่วนใหญ่จะเป็นวิธีการที่มีแรงบันดาลใจมาจากการเรียนแบบหลักการของธรรมชาติและนำมาดัดแปลงเป็นอัลกอริทึมเพื่อใช้ในการแก้ปัญหาต่าง ๆ ตัวอย่างของ metaheuristic algorithms เช่น genetic algorithm, evolutionary computation, simulated annealing, tabu search เป็นต้น
\subsection{Genetic algorithms}
Genetic algorithm เป็นหนึ่งใน metaheuristic ซึ่งเป็นเทคนิคสำหรับค้นหาผลลัพธ์หรือคำตอบโดยประมาณของปัญหา โดยอาศัยหลักการจากทฤษฎีวิวัฒนาการทางชีววิทยาและหลักการคัดเลือกตามธรรมชาติ 
\linebreak กล่าวคือ สิ่งมีชีวิตที่เหมาะสมที่สุดจึงจะอยู่รอด โดยกระบวนการคัดเลือกได้เปลี่ยนแปลงสิ่งมีชีวิตให้เหมาะสมยิ่งขึ้นด้วยตัวดำเนินการทางพันธุกรรม เช่น การสืบพันธุ์ การแลกเปลี่ยนยีน การกลายพันธุ์ เป็นต้น โดยขั้นตอนการทำงานของ genetic algorithm มีดังนี้ 
\begin{enumerate}
  \item initial population เป็นขั้นตอนเริ่มต้นของอัลกอริทึมซึ่งจะทำการกำหนดชุดข้อมูลผลลัพธ์ เรียกชุดข้อมูลนี้ว่า population ซึ่งชุดข้อมูลนี้จะประกอบไปด้วยผลลัพธ์ที่ถูกเข้ารหัสในรูปแบบของสายอักขระที่แต่ละอักขระเป็นบิต 0 หรือ 1 เรียกแต่ละบิตนี้ว่า gene
  โดยที่แต่ละสายอักขระที่ประกอบจาก genes นี้เรียกว่า chromosome โดยชุดข้อมูลนี้อาจจะสุ่มแต่ละ gene ขึ้นมาเป็นค่าเริ่มต้น
  \item fitness function เป็น function สำหรับใช้ในการคัดเลือกชุดข้อมูลที่เหมาะสมให้สามารถอยู่ต่อไปได้ โดยจะมีการคำนวนค่า fitness scores ให้กับแต่ละ chromosome
  โดย fitness scores จะขึ้นอยู่กับความพอใจในผลลัพธ์ที่ได้ของผู้พัฒนา
  \item genetic operator คือวิธีการในการปรับเปลี่ยนรูปแบบโครงสร้างของ chromosome ที่เหมาะสมสำหรับรุ่นถัดไปของกระบวนการ ซึ่งมีวิธีการอยู่ 3 แบบหลัก ๆ ได้แก่
  \begin{itemize}
  \item selection เป็นการเลือกคู่ chromosome ของข้อมูลที่เหมาะสม เพื่อให้ chromosome คู่นี้ส่งต่อ gene ที่ดีแล้วไปยัง chromosome รุ่นถัดไป โดยจะเลือก chromosome ที่มีค่า fitness scores มากที่สุด
  \item crossover เป็นการสุ่มเลือกตำแหน่งระหว่าง genes จาก parent chromosome 1 คู่ โดย genes ด้านซ้ายหรือขวาของจุดแบ่งจะถูกสับเปลี่ยนกันระหว่าง parent chromosome คู่นั้น
  \item mutation เป็นการสุ่มกลับค่าของ gene ใน chromosome ให้มีค่าตรงกันข้ามโดยมีค่าความน่าจะเป็นต่ำ ๆ เพื่อป้องกันไม่ให้ผลลัพธ์ converge ก่อนที่ควรจะเป็น
\end{itemize}
\end{enumerate}
genetic algorithm สามารถจบการทำงานได้หลายวิธี โดยมีเงื่อนไขในการจบการทำงานดังนี้
\begin{itemize}
  \item จบการทำงานเมื่อ population เปลี่ยนผ่านไปถึงรุ่นที่ต้องการแล้ว 
  \item จบการทำงานหาก population ไม่มีการพัฒนาแล้ว หรือไม่มีการเปลี่ยนแปลงไปในทางที่ดีขึ้นเป็นระยะเวลาหนึ่ง
  \item จบการทำงานเมื่อ fitness scores ของ population มีค่าเท่าที่ต้องการแล้ว
\end{itemize}
\section{local search}
local search เป็นวิธีที่ใช่สำหรับค้นหาคำตอบรูปแบบหนึ่ง 


\section{\ifcpe%
ความรู้ตามหลักสูตรซึ่งถูกนำมาใช้หรือบูรณาการในโครงงาน
\else%
ISNE knowledge used, applied, or integrated in this project
\fi
}
\begin{itemize}
  \item 261218 algorithm for computer engineering ได้นำวิธีดำเนินงาน หลักการและทฤษฏี ดังนี้มาใช้เพื่อแก้ไขปัญหาในโครงงานนี้  
  \begin{itemize}
  \item วิธีการคิดและวิเคราะห์ปัญหา
  \item วิธีการแปลงปัญหาใหญ่ที่แก้ไขยากให้กลายเป็นปัญหาที่เล็กกว่าเพื่อใช้วิธีการที่มีอยู่แล้วในการแก้ไขปัญหานั้น
  \item ทฤษฏี graph coloring
  \item การแก้ปัญหา NP-complete และ NP-Hard
  \end{itemize}
\end{itemize}

\section{\ifcpe%
ความรู้นอกหลักสูตรซึ่งถูกนำมาใช้หรือบูรณาการในโครงงาน
\else%
Extracurricular knowledge used, applied, or integrated in this project
\fi
}
ความรู้นอกหลักสูตรที่ใช้สำหรับการแก้ไขปัญหาของโครงงานเพื่อให้ได้ผลลัพธ์ที่เหมาะสุดที่สุด เราได้ทำการศึกษา หลักการและทฤษฏี ดังนี้
\begin{itemize}
  \item Meta­heuristic 
  \item Genetic algorithm
  \item Local search
\end{itemize}

\chapter{\ifproject%
\ifcpe โครงสร้างและขั้นตอนการทำงาน\else Project Structure and Methodology\fi
\else%
\ifcpe โครงสร้างของโครงงาน\else Project Structure\fi
\fi
}

ในบทนี้จะกล่าวถึงหลักการ และการออกแบบระบบ

\makeatletter

% \renewcommand\section{\@startsection {section}{1}{\z@}%
%                                    {13.5ex \@plus -1ex \@minus -.2ex}%
%                                    {2.3ex \@plus.2ex}%
%                                    {\normalfont\large\bfseries}}

\makeatother
%\vspace{2ex}
% \titleformat{\section}{\normalfont\bfseries}{\thesection}{1em}{}
% \titlespacing*{\section}{0pt}{10ex}{0pt}

\section{ผลการสำรวจของนักศึกเกี่ยวกับตารางสอบปลายภาค}
จากการทำแบบสอบถามเกี่ยวกับตารางสอบปลายภาคของมหาวิทยาลัยชียงใหม่ โดยขอความร่วมมือนักศึกษาทุกระดับชั้นในมหาวิทยาลัยเชียงใหม่ 
ทั้งนักศึกษาที่กำลังศึกษาอยู่รวมทั้งที่สำเร็จการศึกษาไปแล้ว เพื่อสอบถามความคิดเห็นเกี่ยวกับ ข้อดี ข้อเสีย ของตารางสอบที่แต่ละคนได้รับ 
ร่วมถึงปัญหาที่เคยกระทบมา โดยเราสามารถบอกได้ว่าผู้ทำแบบสอบถามส่วนใหญ่นั้นตรวจสอบดูช่วงเวลาสอบของแต่ละวิชาก่อนที่จะลงทะเบียนอย่างสม่ำเสมอ 
แต่ยังมีผู้ทำแบบสอบถามบางส่วนนั้นตรวจสอบดูช่วงเวลาสอบก่อนจะลงทะเบียนเรียนในบางวิชาไม่สม่ำ และมีผู้ทำแบบสอบถามส่วนน้อยที่ไม่มีเคยตรวจสอบปลายภาคอยู่เลยดังรูปที่ \ref{fig:enroll}
ต่อมาเราได้สอบถามเกี่ยวกับการจัดตารางสอบของสำนักทะเบียนมหาลัยเชียงใหม่ได้ข้อผลสรุปว่าผู้ทำแบบสอบถามส่วนใหญ่นั้นไม่ทราบถึงเวลาการจัดตารางสอบของสำนักทะเบียน มีคนที่ทราบเป็นส่วนน้อยเท่าน้ันที่ทราบถึงวิธีการจัดตารางสอบดังรูปที่ \ref{fig:Create_exam} 
่ต่อมาเราได้สอบถามเวลาที่ผู้ทำแบบสอบถามต้องการจะสอบคือช่วงเวลาใดของสัปดาห์โดยได้ผลลัพธ์ดังรูป \ref{fig:time_slot} โดย 7 ลำดับช่วงเวลาที่ผู้ทำแบบสอบถามอยากสอบมากที่สุดเป็นดังนี้(จากมากไปน้อยตามลำดับ)
\begin{enumerate}
  \item สัปดาห์ที่หนึ่ง เวลา 12.00-15.00น. วันศุกร์ 
  \item สัปดาห์ที่หนึ่ง เวลา 12.00-15.00น. วันจันทร์
  \item สัปดาห์ที่หนึ่ง เวลา 12.00-15.00น. วันพุธ
  \item สัปดาห์ที่สอง เวลา 12.00-15.00น. วันอังคาร
  \item สัปดาห์ที่หนึ่ง เวลา 12.00-15.00น. วันอาทิตย์
  \item สัปดาห์ที่สอง เวลา 12.00-15.00น. วันพฤหัสบดี
  \item สัปดาห์ที่สอง เวลา 12.00-15.00น. วันเสาร์
\end{enumerate}

จากข้อมูลเราสามารถบอกได้ว่าผู้ทำแบบสอบถามส่วนใหญ่ต้องการที่จะสอบในแต่ละวันคือช่วงเวลา 12.00-15.00น. ซึ่งผู้ทำแบบสอบถามส่วนมากต้องการจะสอบช่วงเวลานี้มากกว่า 15.30-18.00น. และ 08.00-11.00น. ตามลำดับ ดังรูป \ref{fig:time} 
ุ้และถ้าเราทำการแยกช่วงเวลาที่ผู้ตอบแบบสอบถามต้องการสอบออกเป็นวัน เรายังสามารถบอกได้ว่าผู้ทำแบบสอบถามส่วนใหญ่ต้องการจะสอบต้องการที่จะสอบหนึ่งวันเว้นหนึ่งวันเพื่อที่จะได้มีเวลาในการอ่านหนังสือเตรียมสอบสำหรับวิชาในวันต่อไปมากกว่าสอบติดกัน 
ยังสามารถบอกเพิ่มเติมได้อีกว่าผู้ทำแบบสอบถามส่วนใหญ่ต้องการสอบในช่วงสัปดาห์แรกของช่วงการสอบมากกว่าช่วงสัปดาห์สุดของการสอบเพื่อที่จะมีเวลาพักผ่อนหลังจากทำการสอบทั้งหมดดังรูปที่ \ref{fig:day}

\begin{figure}
\begin{center}
\includegraphics{images/check_enrollment.png}\\[2ex]
\includegraphics{images/type_check_enrollment.png}
\end{center}
\caption[Poem]{จำนวนผู้ตอบแบบสอบถามที่ตรวจสอบตารางสอบปลายภาคก่อนการลงทะเบียน}
\label{fig:enroll}     
\end{figure}

\begin{figure}
\begin{center}
\includegraphics{images/Create_exam.png}\\[2ex]
\includegraphics{images/type_Create_exam.png}
\end{center}
\caption[Poem]{จำนวนผู้ตอบแบบสอบถามที่ทราบวิธีการจัดตารางสอบปลายภาคของสำนักทะเบียน}
\label{fig:Create_exam}     
\end{figure}

\begin{figure}
\begin{center}
\includegraphics[width=\linewidth]{images/bar_chart.png}
\end{center}
\caption[Poem]{ความต้องการในสอบของผู้ตอบแบบสอบถามในแต่ละช่วงเวลา}
\label{fig:time_slot}     
\end{figure}

\begin{figure}
\begin{center}
\includegraphics[width=\linewidth]{images/chart.png}
\end{center}
\caption[Poem]{ความต้องการในสอบของผู้ตอบแบบสอบถามในแต่ละวัน}
\label{fig:day}     
\end{figure}

\begin{figure}
\begin{center}
\includegraphics[width=\linewidth]{images/pie.png}
\end{center}
\caption[Poem]{ความต้องการในสอบของผู้ตอบแบบสอบถามในแต่ละเวลา}
\label{fig:time}     
\end{figure}



\chapter{\ifproject%
\ifcpe การทดลองและผลลัพธ์\else Experimentation and Results\fi
\else%
\ifcpe การประเมินระบบ\else System Evaluation\fi
\fi}

ในบทนี้จะทดสอบเกี่ยวกับการทำงานในฟังก์ชันหลักๆ

\section{การประเมินระบบโดยการวัดค่า Penalty}
ในการประเมินผลระบบจัดตารางสอบที่จะพัฒนาขึ้นมานั้นจะมีการประเมินความสมดุลและความเหมาะสมของตารางสอบที่เป็นผลลัพธ์ของระบบ จากค่าเฉลี่ยของ Penalty 
โดยค่า Penalty นั้นจะคำนวณมาจากตารางสอบของนักศึกษาแต่ละคน
โดยจะมีการนำค่าเฉลี่ย Penalty มาเปรียบเทียบกันระหว่างตารางสอบทั้ง 2 แบบ
ได้แก่
\begin{itemize}
    \item ตารางสอบที่เป็นผลลัพธ์จากระบบ
    \item ตารางสอบดั้งเดิมจากสำนักทะเบียน
\end{itemize}

\subsection{การคิด Penalty}

\section{การประเมินระบบโดยสอบถามความพึงพอใจของนักศึกษา}
\ifproject
\chapter{\ifcpe บทสรุปและข้อเสนอแนะ\else Conclusions and Discussion\fi}

\section{\ifcpe สรุปผล\else Conclusions\fi}

นศ. ควรสรุปถึงข้อจำกัดของระบบในด้านต่างๆ ที่ระบบมีในเนื้อหาส่วนนี้ด้วย

\section{\ifcpe ปัญหาที่พบและแนวทางการแก้ไข\else Challenges\fi}

ในการทำโครงงานนี้ พบว่าเกิดปัญหาหลักๆ ดังนี้

\section{\ifcpe%
ข้อเสนอแนะและแนวทางการพัฒนาต่อ
\else%
Suggestions and further improvements
\fi
}

ข้อเสนอแนะเพื่อพัฒนาโครงงานนี้ต่อไป มีดังนี้

\fi

\bibliography{sampleReport}

\ifproject
\appendix
\chapter{ข้อมูลอื่น ๆ ที่เกี่ยวข้องในการทำโครงงาน}
\newcommand\str[1]{\texttt{#1}}
ในมหาวิทยาลัยเชียงใหม่ มีการแบ่งคณะต่าง ๆ ที่เปิดสอนหลักสูตรปริญญาตรี ดังตารางที่ \ref{tab:faculty_code}

\begin{table}[h]
\setstretch{1.35}
  \centering
  \begin{tabular}{@{}cl@{}}
  \toprule
  \multicolumn{1}{c}{\textbf{รหัสคณะ}} & \multicolumn{1}{c}{\textbf{รายชื่อคณะ}}                 \\ \midrule
  01      & คณะมนุษยศาสตร์                 \\
  02      & คณะศึกษาศาสตร์                 \\
  03      & คณะวิจิตรศิลป์                 \\
  04      & คณะสังคมศาสตร์                 \\
  05      & คณะวิทยาศาสตร์                 \\
  06      & คณะวิศวกรรมศาสตร์              \\
  07      & คณะแพทยศาสตร์                  \\
  08      & คณะเกษตรศาสตร์                 \\
  09      & คณะทันตแพทยศาสตร์              \\
  10      & คณะเภสัชศาสตร์                 \\
  11      & คณะเทคนิคการแพทย์              \\
  12      & คณะพยาบาลศาสตร์                \\
  13      & คณะอุตสาหกรรมเกษตร             \\
  14      & คณะสัตวแพทยศาสตร์              \\
  15      & คณะบริหารธุรกิจ                \\
  16      & คณะเศรษฐศาสตร์                 \\
  17      & คณะสถาปัตยกรรมศาสตร์           \\
  18      & คณะการสื่อสารมวลชน             \\
  19      & คณะรัฐศาสตร์                   \\
  20      & คณะนิติศาสตร์                  \\
  21      & วิทยาลัยศิลปะ สื่อและเทคโนโลยี      \\ \bottomrule
  \end{tabular}
  \caption[ตารางแสดงรหัสคณะและรายชื่อคณะภายในมหาวิทยาลัยเชียงใหม่]{ตารางแสดงรหัสคณะและรายชื่อคณะภายในมหาวิทยาลัยเชียงใหม่ \cite{cmu-faculties}}
  \label{tab:faculty_code}
\end{table}

\section{ข้อมูลจากการสำรวจตารางสอบแต่ละคณะ}
ข้อมูลอื่น ๆ ที่เกี่ยวข้อง ทั้งที่ได้ใช้งานและไม่ได้ใช้งาน ได้เก็บรวบรวมไว้ที่
\begin{quote}
\small
\url{https://github.com/tomta555/final-exam-sched}
\end{quote}
ซึ่งข้อมูลทั้งหมดจะอยู่ในโฟลเดอร์ \str{project\_resources} โดยจะมีโฟลเดอร์ย่อยที่เก็บข้อมูลต่าง ๆ ดังนี้ 
\begin{itemize}
  \item \str{courses\_conflicts} เก็บไฟล์รายวิชาทั้งหมด และไฟล์คู่รายวิชาที่มีนักศึกษาลงทะเบียนพร้อมกัน ในแต่ละภาคการศึกษา
  \item \str{faculty\_capacity} เก็บไฟล์ความจุห้องสอบแต่ละห้องของแต่ละคณะและอาคารเรียนรวม ที่ใช้จัดสอบในช่วงปกติและช่วงโควิด
  \item \str{questionnaires\_result} เก็บไฟล์สรุปรวมผลการสำรวจความคิดเห็นของนักศึกษาเพื่อประเมินระบบ
  \item \str{faculty\_exam\_courses} เก็บไฟล์ข้อมูลดิบที่เป็นตารางสอบ 3 ปีย้อนหลัง ของแต่ละคณะ
  \item \str{regist} เก็บไฟล์ข้อมูลรายวิชาที่นักศึกษาแต่ละคนลงทะเบียนในแต่ละภาคการศึกษา
  \item \str{registration\_exam\_table} เก็บไฟล์ตารางสอบของสำนักทะเบียนฯ ในแต่ละภาคการศึกษา
  \item \str{solution\_exam\_table} เก็บไฟล์ตารางสอบที่ได้จากโปรแกรมในแต่ละภาคการศึกษา แยกตามจำนวน slots ที่จัดสอบ
  \item \str{special\_exam} เก็บไฟล์ที่บอกรายวิชาที่จัดสอบแบบ special exam โดยสำนักทะเบียนฯ
  \item \str{time\_TBA\_courses} เก็บไฟล์ที่บอกรายวิชาที่ไม่ระบุวันหรือเวลาเรียนที่เก็บข้อมูลได้จากเว็บสำนักทะเบียนฯ
\end{itemize}

\noindent ข้อมูลจำนวนความจุห้องสอบรวม แยกตามคณะและอาคารเรียนรวม (ข้อมูลล่าสุด ณ วันที่ 29 มกราคม 2564) แสดงดังตารางที่ \ref{tab:faculty_capa}

\begin{table}[h]
  \setstretch{1.35}
  \centering
  \begin{tabular}{@{}lc@{}}
  \toprule
  \multicolumn{1}{c}{\textbf{รายชื่อคณะ}} & \multicolumn{1}{c}{\textbf{จำนวนความจุที่นั่งสอบรวม}} \\ \midrule
  คณะมนุษยศาสตร์                          & 2310                                                  \\
  คณะศึกษาศาสตร์                          & 1506                                                  \\
  คณะวิจิตรศิลป์                          & 350                                                   \\
  คณะสังคมศาสตร์                          & 800                                                   \\
  คณะวิทยาศาสตร์                          & 3495                                                  \\
  คณะวิศวกรรมศาสตร์                       & 1605                                                  \\
  คณะแพทยศาสตร์                           & 3265                                                  \\
  คณะเกษตรศาสตร์                          & 840                                                   \\
  คณะทันตแพทยศาสตร์                       & 346                                                   \\
  คณะเภสัชศาสตร์                          & 574                                                   \\
  คณะเทคนิคการแพทย์                       & 486                                                   \\
  คณะพยาบาลศาสตร์                         & 1202                                                  \\
  คณะอุตสาหกรรมเกษตร                      & 974                                                   \\
  คณะสัตวแพทยศาสตร์                       & 210                                                   \\
  คณะบริหารธุรกิจ                         & 1191                                                  \\
  คณะเศรษฐศาสตร์                          & 935                                                   \\
  คณะสถาปัตยกรรมศาสตร์                    & 968                                                   \\
  คณะการสื่อสารมวลชน                      & 512                                                   \\
  คณะรัฐศาสตร์                            & 900                                                   \\
  คณะนิติศาสตร์                           & 504                                                   \\
  วิทยาลัยศิลปะ สื่อและเทคโนโลยี          & 365                                                   \\
  อาคารเรียนรวม                           & 3015                                                  \\ \bottomrule
  \end{tabular}
  \caption{ตารางแสดงจำนวนความจุห้องสอบแยกตามคณะและอาคารเรียนรวม}
  \label{tab:faculty_capa}
\end{table}

\chapter{\ifcpe คู่มือการใช้งานระบบ\else Manual\fi}
\section{โครงสร้างไดเรกทอรีสำหรับข้อมูลนำเข้าของโปรแกรม}
\label{apd:snd_apd}
ข้อมูลนำเข้าของโปรแกรมต้องมีการจัดเก็บตามโครงสร้างไดเรกทอรี่และมีการตั้งชื่อไฟล์ดังที่แสดงด้านล่าง

\begin{forest}
  for tree={
    font=\ttfamily,
    grow'=0,
    child anchor=west,
    parent anchor=south,
    anchor=west,
    calign=first,
    inner xsep=7pt,
    edge path={
      \noexpand\path [draw, \forestoption{edge}]
      (!u.south west) +(7.5pt,0) |- (.child anchor) pic {folder} \forestoption{edge label};
    },
    % style for your file node 
    file/.style={edge path={\noexpand\path [draw, \forestoption{edge}]
      (!u.south west) +(7.5pt,0) |- (.child anchor) \forestoption{edge label};},
      inner xsep=2pt,font=\small\ttfamily
                 },
    before typesetting nodes={
      if n=1
        {insert before={[,phantom]}}
        {}
    },
    fit=band,
    before computing xy={l=15pt},
  } 
[root-folder
  [data
    [exam-courses-faculty
        [01.in,file]
        [02.in,file]
        [03.in,file]
        [...,file]
        [19.in,file]
        [20.in,file]
        [21.in,file]
    ]
    [all-exam-courses.in,file]
    [conflicts.in,file]
    [enrolled-courses.in,file] 
    [faculty-capacity.in,file]
    [regist.in,file]
    [regist-studentid.in,file]
  ]
  [final\_exam\_graph\_coloring.py,file]
  [penalty\_calc.py,file]
  [start\_penalty\_report.py,file]
  [start\_scheduler.py,file]
  [std\_data\_to\_json.py,file]
]
\end{forest}
\begin{figure}
    \begin{center}
      \includegraphics[]{images/all_exam1.png}
    \end{center}
    \caption[ตัวอย่างไฟล์รายวิชาที่มีสอบ]{ตัวอย่างไฟล์รายวิชาที่มีสอบ}
    \label{fig:all_courses}     
\end{figure}
\begin{figure}
    \begin{center}
    \includegraphics[]{images/capacity1.png}
    \end{center}
    \caption[ตัวอย่างไฟล์ความจุห้องสอบ]{ตัวอย่างไฟล์ความจุห้องสอบ}
    \label{fig:capacity}     
\end{figure}
\begin{figure}
    \begin{center}
      \includegraphics[]{images/conflicts1.png}
    \end{center}
    \caption[ตัวอย่างไฟล์คู่วิชาที่มีนักศึกษาลงทะเบียนพร้อมกัน]{ตัวอย่างไฟล์คู่วิชาที่มีนักศึกษาลงทะเบียนพร้อมกัน}
    \label{fig:conflicts}     
\end{figure}
\begin{figure}
    \begin{center}
      \includegraphics[]{images/courses1.png}
    \end{center}
    \caption[ตัวอย่างไฟล์วิชาที่มีนักศึกษาลงทะเบียน]{ตัวอย่างไฟล์วิชาที่มีนักศึกษาลงทะเบียน}
    \label{fig:courses}     
\end{figure}
\begin{figure}
    \begin{center}
      \includegraphics[]{images/regist1.png}
    \end{center}
    \caption[ตัวอย่างไฟล์ลงทะเบียนของนักศึกษา]{ตัวอย่างไฟล์ลงทะเบียนของนักศึกษา}
    \label{fig:regist}     
\end{figure}
\begin{figure}
    \begin{center}
      \includegraphics[width=\linewidth]{images/regist_hashed.png}
    \end{center}
    \caption[ตัวอย่างไฟล์ลงทะเบียนของนักศึกษาที่มีรหัสนักศึกษา]{ตัวอย่างไฟล์ลงทะเบียนของนักศึกษาที่มีรหัสนักศึกษา}
    \label{fig:regist_hashed}     
\end{figure}


\begin{itemize}
  \item ในโฟลเดอร์ \str{data} ประกอบไปด้วยไฟล์ \str{all-exam-course.in}, \str{faculty-capacity.in},
  \\ \str{conflicts.in}, \str{enrolled-courses.in}, \str{regist.in}, \str{regist-studentid.in} รวมทั้งหมดจำนวน 6 ไฟล์ แต่ละไฟล์ เป็นไฟล์ text 
  \item ในไฟล์ \str{all-exam-course.in} แต่ละบรรทัดประกอบด้วย <รหัสวิชาที่มีการจัดสอบ> หนึ่งบรรทัดต่อหนึ่งรหัสวิชา ตัวอย่างดังรูปที่~\ref{fig:all_courses}
  \item ในไฟล์ \str{faculty-capacity.in} แต่ละบรรทัดประกอบด้วย รหัสคณะและจำนวนความจุรวมของห้องสอบของคณะนั้น ๆ มีรูปแบบดังนี้ <รหัสคณะ> <ความจุห้องสอบคณะ> แต่ละส่วนคั่นด้วย เว้นวรรค (space) หนึ่งบรรทัดต่อหนึ่งคณะ ตัวอย่างดังรูปที่~\ref{fig:capacity}
  \item ในไฟล์ \str{conflicts.in} แต่ละบรรทัดประกอบด้วย รหัสวิชาสองวิชาที่มีนักศึกษาลงทะเบียนพร้อมกันในภาคการศึกษานั้น ๆ และ จำนวนนักศึกษาที่ลงทะเบียนคู่วิชานี้ มีรูปแบบดังนี้ <รหัสวิชา> <รหัสวิชา> <จำนวนนักศึกษา> แต่ละส่วนคั่นด้วย เว้นวรรค (space) ตัวอย่างดังรูปที่~\ref{fig:conflicts}
  \item ในไฟล์ \str{enrolled-courses.in} แต่ละบรรทัดประกอบด้วย รหัสวิชาและจำนวนนักศึกษาที่ลงทะเบียนในวิชานั้น ๆ มีรูปแบบดังนี้ <รหัสวิชา> <จำนวนนักศึกษา> แต่ละส่วนคั่นด้วย เว้นวรรค (space) ตัวอย่างดังรูปที่~\ref{fig:courses}
  \item ในโฟลเดอร์ \str{data/exam-courses-faculty} ประกอบไปด้วยไฟล์ 01.in ถึง 21.in ซึ่งเป็นไฟล์ text โดยตั้งชื่อไฟล์ตามรหัสคณะ ในแต่ละไฟล์ประกอบด้วย รหัสวิชาที่มีการจัดสอบ หนึ่งบรรทัดต่อหนึ่งรหัสวิชา ตัวอย่างดังรูปที่~\ref{fig:all_courses} 
  \item ในไฟล์ \str{regist.in} แต่ละบรรทัดประกอบด้วย ข้อมูลลงทะเบียนของนักศึกษาหนึ่งคน ซึ่งประกอบด้วยด้วยรหัสวิชาทั้งหมดที่นักศึกษาคนนั้นลงทะเบียน มีรูปแบบดังนี้ <รหัสวิชา> <รหัสวิชา> <รหัสวิชา> ... แต่ละวิชาคั่นด้วย เว้นวรรค (space) ตัวอย่างดังรูปที่~\ref{fig:regist}
  \item ในไฟล์ \str{regist-studentid.in} แต่ละบรรทัดประกอบด้วย ข้อมูลลงทะเบียนของนักศึกษาหนึ่งคน \\ ประกอบด้วยด้วยรหัสนักศึกษาและวิชาทั้งหมดที่นักศึกษาคนนั้นลงทะเบียน มีรูปแบบดังนี้ \\ <รหัสนักศึกษา> <รหัสวิชา> <รหัสวิชา> <รหัสวิชา> ... แต่ละส่วนคั่นด้วย เว้นวรรค (space) \\ ตัวอย่างดังรูปที่~\ref{fig:regist_hashed}
  โดยในตัวอย่างได้มีการแฮช (hash) รหัสนักศึกษา เพื่อปกป้องข้อมูลส่วนตัวของนักศึกษา
\end{itemize}

\section{การติดตั้ง Python ก่อนการใช้งานโปรแกรม}

\CIreply{ใช้ teletype font สำหรับส่วนที่เกี่ยวข้องกับ command line}
\paragraph{การติดตั้ง Python}
\begin{enumerate}
  \item ไปยัง \url{https://www.python.org/downloads/} เพื่อดาวน์โหลดตัว setup ของ Python
  \item เมื่อเปิดไฟล์ setup ให้ติ๊กถูกที่ช่อง Add python 3.x to PATH ด้วย จากนั้น install ตามปกติ
  \item เมื่อติดตั้ง Python เสร็จแล้ว ให้ทดลองตรวจสอบว่า Python นั้นติดตั้งสำเร็จ โดยการเปิด cmd หรือ powershell แล้วใช้คำสั่ง 
  \begin{verbatim}
    py --version
  \end{verbatim}
  จะแสดง version ของ Python ที่ติดตั้งลงในเครื่อง
\end{enumerate}

\paragraph{การติดตั้งแพคเกจ \str{networkx}}
\begin{enumerate}
  \item เมื่อติดตั้ง Python สำเร็จแล้ว ให้เปิด cmd หรือ powershell จากนั้นใช้คำสั่ง
  \begin{verbatim}
    pip install networkx
  \end{verbatim}
  \item หากมีการยืนยันให้กด \verb+y+ และ enter เพื่อทำการติดตั้งแพคเกจ
  \item เมื่อติดตั้งแพคเกจเรียบร้อยแล้ว ถือว่าเสร็จสิ้นการเตรียมการ สามารถเรียกใช้งานโปรแกรมจัดตารางสอบได้ทันที
\end{enumerate}

\section{วิธีการใช้งานโปรแกรม}
\begin{enumerate}
    \item จัดเตรียมไฟล์ข้อมูลนำเข้าต่าง ๆ ตามที่ได้ระบุไว้ใน \ref{apd:snd_apd} ให้เรียบร้อย
    \item เปิดไฟล์ \verb+final_exam_graph_coloring.py+
    \item ทำการแก้ไข path ของไฟล์ข้อมูลนำเข้าต่าง ๆ ให้ตรงตามที่กำหนด หากได้ตั้งชื่อไฟล์ และจัดแยกไฟล์ต่าง ๆ ไว้ตามโฟลเดอร์ที่กำหนดแล้วไม่จำเป็นต้องแก้ไขตัวแปร path ใด ๆ
    \item ทำการแก้ไขจำนวน slot ที่ใช้สอบ โดยแก้ไขที่ตัวแปร \verb+TOTAL_SLOTS+
\end{enumerate}
ในกรณีที่ต้องการจัดตารางสอบด้วยอัลกอริทึมเพียงวิธีเดียว สามารถเรียกใช้งานโปรแกรมผ่าน console หรือ terminal เช่น cmd หรือ powershell ได้ด้วยคำสั่ง 
\begin{verbatim}
    py final_exam_graph_coloring.py [OPTION]
\end{verbatim}
ซึ่ง \verb+OPTION+ เป็นไปได้ 4 แบบ ได้แก่ [-deg, -std, -deg-bfs, -std-bfs]

\begin{verbatim}
    ตัวอย่างการเรียกใช้งานโปรแกรม
    py final_exam_graph_coloring.py -deg
\end{verbatim}

\noindent ในกรณีที่ต้องการจัดตารางสอบทั้งหมด 4 วิธี สามารถเรียกใช้งานโปรแกรมผ่าน console หรือ terminal เช่น cmd หรือ powershell ได้ด้วยคำสั่ง
\begin{verbatim}
    py start_scheduler.py
\end{verbatim}
\section{วิธีการใช้งานโปรแกรมคำนวณค่า penalty}
\begin{enumerate}
    \item จัดเตรียมไฟล์ข้อมูลนำเข้าต่าง ๆ ตามที่ได้ระบุไว้ใน \ref{apd:snd_apd} ให้เรียบร้อย
    \item เปิดไฟล์ \verb+penalty_calc.py+
    \item ทำการแก้ไข path ของไฟล์ข้อมูลนำเข้าต่าง ๆ ให้ตรงตามที่กำหนด หากได้ตั้งชื่อไฟล์ และจัดแยกไฟล์ต่าง ๆ ไว้ตามโฟลเดอร์ที่กำหนดแล้วไม่จำเป็นต้องแก้ไขตัวแปร path ใด ๆ
    \item ทำการแก้ไขจำนวน slot ที่ใช้สอบให้ตรงกับ solution ของตารางสอบที่จัดโดยโปรแกรม โดยแก้ไขที่ตัวแปร \verb+TOTAL_SLOTS+
\end{enumerate}
ในกรณีที่ต้องการคิดคำนวณค่า penalty ของตารางสอบเพียง 1 solution สามารถเรียกใช้งานโปรแกรมผ่าน console หรือ terminal เช่น cmd หรือ powershell ได้ด้วยคำสั่ง 
\begin{verbatim}
    py penalty_calc.py <solution_file>
\end{verbatim}
โดยที่ \verb+<solution_file>+ คือ path ของไฟล์ solution ตารางสอบที่ได้จากโปรแกรมจัดตารางสอบ

\begin{verbatim}
    ตัวอย่างการเรียกใช้งานโปรแกรมคำนวณค่า penalty
    py penalty_calc.py solution/graph-coloring-solution-deg.txt
\end{verbatim}

\noindent ในกรณีที่ต้องการคิดคำนวณค่า penalty ของตารางสอบทั้งหมด 4 วิธี สามารถเรียกใช้งานโปรแกรมคำนวณค่า penalty ผ่าน console หรือ terminal เช่น cmd หรือ powershell ได้ด้วยคำสั่ง
\begin{verbatim}
    py start_penalty_report.py
\end{verbatim}
โดยให้เปิดไฟล์ \verb+start_penalty_report.py+ เพื่อแก้ไขตัวแปร \verb+solution_folder+ ให้ชี้ไปยังโฟลเดอร์ที่เก็บ solution ตารางสอบทั้งหมดที่เป็น output ของโปรแกรมจัดตารางสอบให้ถูกต้องก่อนการเรียกใช้งานคำสั่งด้านบน

%% Display glossary (optional) -- need glossary option.
\ifglossary\glossarypage\fi

%% Display index (optional) -- need idx option.
\ifindex\indexpage\fi

\begin{biosketch}
\begin{center}
  \includegraphics[width=1.5in]{mugshot.jpg}
\end{center}
\begin{itemize}[label={},leftmargin=*]
  \item ชื่อ-นามสกุล: นาย ธนวงศ์ เสนีวงศ์ ณ อยุธยา
  \item ระดับการศึกษา: ปริญญาตรี สาขา วิศวกรรมคอมพิวเตอร์ ภาควิชา วิศวกรรมคอมพิวเตอร์
  \item คณะ วิศวกรรมศาสตร์ มหาวิทยาลัยเชียงใหม่
  \item มือถือ: 0899577355 
  \item E-mail: thanawong.saneewong@gmail.com
\end{itemize}


\noindent \textbf{ผลงานและกิจกรรมต่าง ๆ ที่ได้เข้าร่วม}
\begin{itemize}
  \item เข้าร่วมฝึกอบรมค่ายโอลิมปิกวิชาการ สอวน. สาขาคอมพิวเตอร์ ค่าย 1 ปีการศึกษา 2557
  \item เข้าร่วมฝึกอบรมค่ายโอลิมปิกวิชาการ สอวน. สาขาคอมพิวเตอร์ ค่าย 2 ปีการศึกษา 2557
  \item เข้าร่วมฝึกอบรมค่ายโอลิมปิกวิชาการ สอวน. สาขาคอมพิวเตอร์ ค่าย 1 ปีการศึกษา 2558
  \item เป็น staff และผู้ช่วยกรรมการตัดสินในงานแข่งขัน iCode Programming Contest 2018
  \item ผ่านเข้ารอบนำเสนอผลงานและได้รับเงินสนับสนุนการทำโครงงาน ในการแข่งขัน NSC 2021\\
  ที่จัดขึ้นโดย ศูนย์เทคโนโลยีอิเล็กทรอนิกส์และคอมพิวเตอร์แห่งชาติ (NECTEC)
\end{itemize}
\newpage
\begin{center}
  \includegraphics[width=1.5in]{mugshot.jpg}
\end{center}
\begin{itemize}[label={},leftmargin=*]
  \item ชื่อ-นามสกุล: นาย กฤษฏิ์ อุปนันท์
  \item ระดับการศึกษา: ปริญญาตรี สาขา วิศวกรรมคอมพิวเตอร์ ภาควิชา วิศวกรรมคอมพิวเตอร์
  \item คณะ วิศวกรรมศาสตร์ มหาวิทยาลัยเชียงใหม่
  \item มือถือ: 0954535187
  \item E-mail: krit.upanun@gmail.com
\end{itemize}


\noindent \textbf{ผลงานและกิจกรรมต่าง ๆ ที่ได้เข้าร่วม}
\begin{itemize}
  \item เข้าร่วมแข่งขันการเขียนโปรแกรมคอมพิวเตอร์ระดับมัธยมศึกษา iCode 2014
  \item รางวัลชมเชย การแข่งขันการเขียนโปรแกรมคอมพิวเตอร์ระดับมัธยมศึกษา iCode 2015
  \item เข้าร่วมฝึกอบรมค่ายโอลิมปิกวิชาการ สอวน. สาขาคอมพิวเตอร์ ค่าย 1 ปีการศึกษา 2558
  \item เข้าร่วมฝึกอบรมค่ายโอลิมปิกวิชาการ สอวน. สาขาคอมพิวเตอร์ ค่าย 2 ปีการศึกษา 2558
  \item ผ่านเข้ารอบนำเสนอผลงานและได้รับเงินสนับสนุนการทำโครงงาน ในการแข่งขัน NSC 2021\\
  ที่จัดขึ้นโดย ศูนย์เทคโนโลยีอิเล็กทรอนิกส์และคอมพิวเตอร์แห่งชาติ (NECTEC)
\end{itemize}

\end{biosketch}
\fi % \ifproject
\end{document}
